% ************************** Thesis Abstract *****************************
\begin{abstract}


Given recent advancements in observational solar physics, both in quality and quantity, the time is right to revise the chromospheric feature, macrospicules.
These jet-like phenomena are larger than their semi-namesake spicules, which only extend to $10$ Mm as an absolute maximum and are ubiquitous in the solar chromosphere, particularly within intergranular lanes.
However, macrospicules are not as large as the so-called coronal jets or the X-ray jets, generally observed in hotter temperature lines and penetrating much higher into the solar atmosphere.
The aim of this work is to better classify macrospicules as a population and to detect any possible relationships; such as relation to the solar cycle, impacts on coronal heating or as a solar wind accelerator, on a global scale.

This is achieved first by means of a statistical sample of macrospicules.
We utilise the first two and a half years of Atmospheric Imaging Assembly on board the Solar Dynamics Observatory's (AIA/SDO) operation window and measure macrospicules properties throughout.
This two and a half year sample acts as a proxy for the ramp from solar minima in mid $2010$ to maxima in late $2012$.
Over this time period we find a general increasing trend for the properties of the macrospicules.
A range of charcteristic features of the macrospicules, such as: maximum length and width, maximum velocity and lifetime are stated and compared to the current literature.
This same sample is then tested against the Carrington longitude investigate a possible relation to what has been termed, an active longitude.
In this case, we find that the macrospicules do have a correlation to the so-called active longitude.

Lastly, this work presents a detailed case study of a macrospicule, utilising a wide range of available imagers.
The case study involves a jet-like feature that is seen at the solar limb in Crisp Imaging Spectropolarimeter at the Swedish Solar Telescope, AIA/SDO and the Extreme Ultra Violet Imager on STEREO (Solar Terrestrial Earth RElations Observatory).
Applying a Markov Chain Monte Carlo method we analyse the spectroscopic data from CRISP and build a profile of the line of sight velocities of the jet.
Lastly, we attempt to determine whether or not the jets have an effect on the atmosphere above it.
\end{abstract}


















