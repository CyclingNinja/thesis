% !TeX root = ../thesis.tex
%*****************************************************************************************
%*********************************** First Chapter ***************************************
%*****************************************************************************************
\label{ch:Intro}
\chapter{Introduction}  %Title of the First Chapter


%%%%%%%%%%%%%%%%%%%%%%%%%%%%%%%%%%%%%%%%%%%%%%%%%%%%%%%%%%% PAPER TEXT %%%%%%%%%%%%%%%%%%%%%%%%%%%%%%%%%%%%%%%%%%%%%%%%%%%%%%%%%%%%%%%%%%%%%%%%%%%%%%



\section{The Sun}
The Sun is an extremely complex ball of plasma, the formation of which generated the collection of planets and asteroids we call the solar system.
As such, the study of our nearest star should be at the forefront of our research into the cosmos, any model we build to examine other stars must first accurately describe our star. 
The Sun takes its place on the Hertzsprung-Russell diagram as an early life main sequence star at the yellow end of the stellar spectrum.
It is a population 3 star, meaning that it has a high metallic content, a fact which we exploit to aid in our observations.

\subsection{Structure}
The sun is approximately $6.96 \times 10^{8}$ m in diameter and has an average density of $1.4$ gcm^{-3}, approximatley $40\%$ more than the density of water and constitutes $98\%$ of the mass within our solar system.

The structure of the Sun can be divided into internal and external.
Internal structure has been inferred by the use of techniques such as seismology, therefore, we still have many questions as to the exact mechanisms dominating below the photosphere.
These examinations have revealed a stratified structure, and the centre of which is the fusion core.
Currently, the fusion process is converting $2$ Hydrogen atoms to $1$ Deuteron, positron and neutrino; this Deuteron then reacts with another proton to form ${^3}He$ and a gamma particle, and lastly two $^{3}He$ combine to form $^{4}He$ and two protons, known as the proton-proton chain. 
However as the star evolves this process will change as one fuel source runs out, the previous product becomes the new fuel.
\emph{E.g.} the next phase would convert two Helium atoms to a Beryllium atom and a gamma particle.
The particles given off in the form of gamma particles and neutrinos carry away the excess energy, and go on to form the radiative zone.
These processes require extremely high pressure and temperature in order to overcome the binding energy of the atoms taking part in the process.
The conditions needed for this fusion process are generated by gravitational pressure exerted by the rest of the star which inherently increases the temperature in accordance with the ideal gas law.

The radiative zone is appropriately named, in that, the excess energy generated in the core radiates through this section.
It has been calculated that a photon emitted in the core takes on average $100,000$ years to move through the radiative zone as a result of the random walk, a process by which a photon is emitted and absorbed repeatedly.

The tachocline is a thin layer between the radiative zone and the convection zone, of which not a great deal is known.
This is the point where p-mode oscillations, used in seismology, cease to penetrate further into the solar interior and is strongly suspected to have a crucial role in generating the solar magnetic field.
There have been recent studies which suggest that the tacholine is responsible for several $1.3$ to $3$ year cycles in feature observed higher in the solar atmosphere, however this has not been definitively proven.



Convection Zone

The photosphere is the layer of the Sun which we can observe with the naked eye.
The photosphere is so called, because this layer emits light in the visible spectrum, and has a temperature range of $6000$ K at the base, and $4700$ K at the top. 
From a distance, the photosphere appears to be a smooth sphere, however upon closer inspection we observe granulation, as a result of the convection zone, is ubiquitous throughout the layer.
Granulation appears as dark lines and bright 'bubbles' expanding and collapsing as hot material rises, cools and consequently falls back down into the solar interior.
These features tend to measure approximately $2$ Mm from one boundary to the other, which, are know as inter-granular lanes.
When granules bulk motion is observed , it becomes apparent that they group into larger, super-granules, where the overall motion of the granules radiates away from a central point until they fade at a point where they meet similar granules from another super-granule.
% need a granulation figure here
The most prominent feature in the photosphere are Sunspots, these are widely documented.
Observations of sunspots are dated as far back as Galileo and his first telescopes.
Sunspots

Chromosphere

Transition Region

Corona

Solar Wind

Heliopause

\subsection{History of observations (generally)}

\section{Plasma behaviour}

\subsection{MHD}

\subsection{Reconnection}

\section{Jet and Macrospicules}

\subsection{The ejecta zoo}

\subsection{Macrospicules}

\subsection{Numerical Simulations}
































\begin{pycode}[chapter1]
from __future__ import print_function

ch1 = texfigure.Manager(pytex, number=1, base_path='./Chapter1/')
\end{pycode}

\begin{pycode}[chapter1]
fig = plt.figure(dpi=100, figsize=texfigure.figsize(pytex, height_ratio=1.5))

xx = np.linspace(0, 6*np.pi, 1000)
plt.plot(xx, np.sin(xx))

fig.tight_layout()
fig.subplots_adjust(bottom=0.15)
photosphere = ch1.save_figure('sin', fig, fext='.pgf')
photosphere.caption = "Some sine wave."
\end{pycode}

\py[chapter1]|photosphere|



Hello world, checkout \cref{fig:sin}.


\begin{pycode}[chapter1]
	
multi = texfigure.MultiFigure(4,1)
width = 0.9

X = [[5,6], [7,8], [5,6], [7,8]]
Y = [[1,2], [3,4], [1,2], [3,4]]

for x,y in zip(X,Y):
	fig = plt.figure(figsize=texfigure.figsize(pytex, scale=width))
	plt.plot(x, y, 'o')
	
	Fig1 = ch1.save_figure('test', fig)
	Fig1.subfig_width=r"{}\columnwidth".format(width)
	multi.append(Fig1)
\end{pycode}

\py[chapter1]|multi[0:2]|
\py[chapter1]|multi[2:4]|