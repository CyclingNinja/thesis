% !TeX root = ../thesis.tex
%*****************************************************************************************
%*********************************** First Chapter ***************************************
%*****************************************************************************************
\label{ch:Intro}
\chapter{Introduction}  %Title of the First Chapter


%%%%%%%%%%%%%%%%%%%%%%%%%%%%%%%%%%%%%%%%%%%%%%%%%%%%%%%%%%% PAPER TEXT %%%%%%%%%%%%%%%%%%%%%%%%%%%%%%%%%%%%%%%%%%%%%%%%%%%%%%%%%%%%%%%%%%%%%%%%%%%%%%



\section{The Sun}
The Sun is an extremely complex ball of plasma, the formation of which generated the collection of planets and asteroids we call the solar system.
As such, the study of our nearest star should be at the forefront of our research into the cosmos, any model we build to examine other stars must first accurately describe our star. 
The Sun takes its place on the Hertzsprung-Russell diagram as an early life main sequence star at the yellow end of the stellar spectrum.
It is a population 3 star, meaning that it has a high metallic content, a fact which we exploit to aid in our observations.

\subsection{Structure}
The sun is approximately $6.96 \times 10^{8}$ m in diameter and has an average density of $1.4$ gcm^{-3}, approximatley $40\%$ more than the density of water and constitutes $98\%$ of the mass within our solar system.

The structure of the Sun can be divided into internal and external.
Internal structure has been inferred by the use of techniques such as seismology, therefore, we still have many questions as to the exact mechanisms dominating below the photosphere.
These examinations have revealed a stratified structure, and the centre of which is the fusion core.
Currently, the fusion process is converting $2$ Hydrogen atoms to $1$ Deuteron, positron and neutrino; this Deuteron then reacts with another proton to form ${^3}He$ and a gamma particle, and lastly two $^{3}He$ combine to form $^{4}He$ and two protons, known as the proton-proton chain. 
However as the star evolves this process will change as one fuel source runs out, the previous product becomes the new fuel.
\emph{E.g.} the next phase would convert two Helium atoms to a Beryllium atom and a gamma particle.
The particles given off in the form of gamma particles and neutrinos carry away the excess energy, and go on to form the radiative zone.
These processes require extremely high pressure and temperature in order to overcome the binding energy of the atoms taking part in the process.
The conditions needed for this fusion process are generated by gravitational pressure exerted by the rest of the star which inherently increases the temperature in accordance with the ideal gas law.

The radiative zone is appropriately named, in that, the excess energy generated in the core radiates through this section.
It has been calculated that a photon emitted in the core takes on average $100,000$ years to move through the radiative zone as a result of the random walk, a process by which a photon is emitted and absorbed repeatedly.

The tachocline is a thin layer between the radiative zone and the convection zone, of which not a great deal is known.
This is the point where p-mode oscillations, used in seismology, cease to penetrate further into the solar interior and is strongly suspected to have a crucial role in generating the solar magnetic field.
There have been recent studies which suggest that the tacholine is responsible for several $1.3$ to $3$ year cycles in feature observed higher in the solar atmosphere, however this has not been definitively proven.
It has also been proposed as the source for a dynamo generating the magnetic field.
The most important result of the tachocline is that at this point, the motion changes from uniform behaviour of the radiation zone to the differential rotation of the convective zone.
This differential rotation is the cause of much of the complexity inherent in the solar body.

% more needed here
Beyond the tachocline, radiative heat transport ceases to be as effective, and as such, convection becomes the primary form energy transport, thus, this region is named the convection zone. 
A result of this lower temperature is a transformation in the bulk motion and ao overall behaviour of the plasma.
Within the radiative zone the plasma is approximately $5$ MK and is consequently fully ionised, however with the decrease in temperature, comes a transition in the behaviour of the plasma, density drops, and the heavier elements are now not ionised.
The process by which convection takes place is cellular in nature, hot plasma rising and cool plasma falling back through the solar interior.
 


The photosphere is the layer of the Sun which we can observe with the naked eye.
The photosphere is so called, because this layer emits light in the visible spectrum, and has a temperature range of $6000$ K at the base, and $4700$ K at the top. 
From a distance, the photosphere appears to be a smooth sphere, however upon closer inspection we observe granulation, as a result of the convection zone, is ubiquitous throughout the layer.
Granulation appears as dark lines and bright 'bubbles' expanding and collapsing as hot material rises, cools and consequently falls back down into the solar interior.
These features tend to measure approximately $2$ Mm from one boundary to the other, which, are know as inter-granular lanes.
When granules bulk motion is observed , it becomes apparent that they group into larger, super-granules, where the overall motion of the granules radiates away from a central point until they fade at a point where they meet similar granules from another super-granule.
% need a granulation figure here

The most prominent feature in the photosphere are Sunspots, these are widely documented.
Observations of sunspots are dated as far back as Galileo and his first telescopes.
These features are significantly cooler that the surrounding photosphere, with a dark centre, known as the umbra, through which open magnetic field emerges from the solar interior.
Spreading away from the umbra is the penumbra, a ring of long thin structures spreading away from the umbra.
These have been widely studied and have become known as fibrils.
The reason these are so widely studied, is that it has become clear that waves travel up these features from the solar interior.
Sunspots are normally part of an extremely complex system of magnetic field, meaning that there are usually many in one active region.
It is not yet clear the exact mechanism by which sunspots are formed, however the favoured hypothesis is that of flux rope emergence.
In this model a 'rope' of magnetic field is forced up through the convection zone and the photosphere carrying plasma frozen into the magnetic field lines with them.
As a result of this formation mechanism, active regions have regions of both negative and positive polarity.
Particularly of note with sunspots, is their continual rotation.
They are observed to rotate in the opposite direction to the rotation of the Sun, while at the same time they migrate to the equator due to the magnetic configuration between the poles.


% Chromosphere
The first observations of the chromosphere were made during total solar eclipses.
When the moon entirely covered the solar disk \emph{i.e.} the photosphere, apparent at the solar limb were colourful streamers radiating from the solar centre.
The density drops sharply from the photosphere to the order of $10{-4}$ and is approximately $2$ Mm thick. 
Over this $2$ Mm layer the temperature rises from $4000$ K to $25,000$ K, this result is currently one of the primary focuses of the solar research community.
The chromosphere is incredibly complex interms of its magnetic structure and the transport of heat, consequently the chromosphere plays and essential role in the formation of explosive events such as solar flares and coronal mass ejections (CMEs).

It is in the chromosphere where we now see the magnetic fields which are rooted in the sunspots we observe in the photosphere.
These appear as large scale loops of plasma which has been locked into the magnetic field lines, extending through this region and right into the corona, they regularly reach 10's of Mm into the atmosphere.
Lower in the chromosphere, we find a second, smaller scale population of loops, and are a few Mm across on average.
Small scale chromospheric loops have been presented as a possible link between the photosphere and the chromosphere, due the their most likely formation being a small scale flux emergence event.
% more needs to be said here

Examining any image of the chromosphere, it is almost impossible to miss the fabled 'forest of spicules'.
Spicules are small, short-lived, explosive events with origins in the chromosphere which extend through the transition region and into the corona.
They are ubiquitous in the chromosphere, their number density being of the order of $10^5$ at any given time, however, their spatial distribution is not uniform.
They are found to form on intergranular lanes, as such their is much debate about how they form.
Candidates for the formation mechanism are; magnetic reconnection, p-mode driven and 'plasma drains', although there is no conclusive evidence for any of these drives.
Spicules high number density has lead many to propose them as a possible solution to the coronal heating problem, be that through instability or propagation of waves into the atmosphere directly.
Recent in depth study has revealed the possibility of two populations of spicules, dubbed, Type-I and Type-II.
Type-I have been defined to be longer lived and less explosive, they are also observed to emerge and have a ballistic motion away from and returning to the Sun.
However, Type-II show significantly more explosive velocities, up to $150$ km/s recorded during their initial formation, however, they are not observed to return and lifetimes are not expected to exceed $5$ mins. 

% need a section on the magnetic network here somewhere

% better citing needed here
Since these initial propositions, however, there has been doubt cast as to whether this is the case or not. 
Cite Zhang 2010 found no statistical separation of populations within spicules and recent publications but the original authors have shown that Type-2 spicules disappear from Ca II and reappear in the hotter Si IV line.
This would imply that spicules are heating as they accelerate through the atmosphere, whether this is because the underlying formation mechanism is different or there is sufficient energy in the initialisation of the spicule to cause heating as they propagate through the atmosphere, has yet to be made clear.

Particularly noticeable in the chromosphere is the appearance of coronal holes.
These appear as regions of dark amongst the bright chromospheric features, this is a result of the cool plasma, lower in the atmosphere becoming visible through these holes.
During the minimum phase of the solar cycle, there are usually two prominent coronal holes at the solar poles. 
These can cover half of the solar disk during particularly inactive solar minima, however, at solar maxima, these polar coronal holes disappear as the magnetic field becomes increasingly complex.
The coronal holes are characterised by open magnetic field lines extending up though the solar atmosphere, whereas, in the quiet Sun (areas not coronal holes) the magnetic field lines are closed, generally forming small and large scale loops.

This magnetic field configuration of open field lines at the poles and profound non-uniformity between is a result of the differential rotation above the tachocline.
Due to faster angular velocities at the equator than at the poles, magnetic field lines which would be straight, pole to pole, are warped in accordance with the frozen in condition.
Eventually, the magnetic field between $60^\circ$ and $-60^\circ$ forms into approximate bands of alternating opposing polarity magnetic field.
The mechanism by which this structure forms, results in converging bands, \emph{i.e.} a sideways 'V' symmetrical around the equator.
This can act as a guide for other solar features, such as sunspots, pushing them towards the solar equator from either side.

A demonstration of this complex magnetic field, is the magnetic bright point.
Theses are prevelent throughout the solar atmosphere, we observe them in the photosphere and corona, however, they are very prevalent in the chromosphere.
They are thought to be regions of very high magnetic pressure, causing higher gas pressure and therefore an increase in temperature.
What would 
% not finished with the chromosphere yet 

There are several large scale features with their roots in the chromosphere.
Filaments, and their off limb counterparts prominences, are very prevalent in solar imaging.
Filaments are observed as dark strands of plasma, having risen from the cool photosphere against the hotter chromosphere, that can extend far across the solar disk and have lifetimes of many months.
Prominences are the same features observed over the limb, and hence appear as bright features against the dark sky.
This allows us to observe the impact on the atmosphere more closely. 
Upon inspection, we observe the plasma falling away from the main body of the prominence, becoming known as coronal rain.
The effect this rain has on the atmosphere is not yet fully understood, but there has be disussion as to its merits in terms of possible heating or cooling effects.

At the top of the chromosphere, temperature of the plasma increases rapidly over a very short distance, approximately $500$ km, this is known at the transition region.
It acts as a barrier, and amplifier, between chromosphere and corona with the temperature rising from $25,000$ K to $2$ MK, the mechanism which causes this is still not understood and is one of the prominent problems in solar physics.

%Corona
Above the transition region we find the corona, high temperature plasma whose behaviour is now dominated by the magnetic field which has emerged from the solar interior.
It is very rich in Iron , which, we can tell is ionised from observations, and such the temperature has a minimum value of $1 \times 10^6$ K.
The corona is a very complex system where features interact strongly with each other.
Here we observe very large scale structures here such as coronal loops and streamers, as well as transient explosive events such as Solar Flares and CME's, regularly in the same event.
Streamers are self descriptive flows of plasma moving radially away from the solar disk, they are particularly prevalent above the coronal holes and over magnetic features, such as loops, the resulting plasma from streamers then go on to contribute to the solar wind.
Coronal loops, CME's and solar flares are tightly related.
As the footpoints of the coronal loops, Sunspots, rotate and migrate towards the equator, the magnetic tension of the loop increases to the point at which a reconnection event is the only way of reducing the magnetic energy in the system.
This mechanism for the formation of solar flares is known as 
The material in the overlying loops becomes unbound as a result of the release of magnetic tension, and is released in the form of a CME, although this is not always the case.

The above model applies more to solar flares which are observed low in the corona and chromosphere, there is a seperate model in which the reconnection occurs much higher in the corona.
There is a standard model for the formation of solar flares.
The models' initial condition involves an arcade of coronal loops, with open magnetic field lines forming a streamer around the outside.
At the top of the coronal loop, a region of cool, dense plasma forms, however, remains suspended by the magnetic field.
Eventually the system will reach a non-equilibrium state and the filament will be ejected outwards from the loop system.
The resulting elongation of the magnetic field, eventually leads to the field lines, previously on opposite sides of the loop, becoming close enough that magnetic force brings them together.
This effectively severs the anchor holding the plasma filament in place, releasing it into the high corona and solar wind. 
This release of energy also has the effect of accelerating particles down the underlying coronal arcade, causing heating and brightening.
The nature of these events is sufficient to cause 'bursts' in the electromagnetic spectrum.
This is the phenomena that has been named the solar flare, which are categorised on a scale by class, B, C, M and X, with X being the highest energy (Wm$^{-2}$) and B being the lowest and 9 subdivisions within each class. 

The remaining ejected material then either falls back to the solar surface or carries on to form a coronal mass ejection.
These are very large scale features in the corona, both spatially and temporally.
They propagate through the corona, into the solar wind and beyond Earth.
They are categorised (in the LASCO database) based upon the appearance, Halo, partial halo and complete.
This is based on the difference with respect to the Earth/viewing position, a CME which is directed at Earth will appear as a halo around the Sun.
Viewed from the side, we see the fine structure of the CME, prevalent is the leading edge, at the fore of the propagating feature beneath which is a void.
The centre of the CME is generally the core, comprising of the filament that has just detached from the solar surface.
They are inherently imbued with the magnetic field that originated in the solar atmosphere.
As this magnetic field propagates through the solar wind, a shock is thought to form at the bow of the feature, these shocks may also excite heavy ions in the solar wind causing turbulence and heating.

The when the coronal materials bulk motion becomes radial, it is defined to be the solar wind.
At this point the plasmsa is dominated by the magnetic field and may be considered collisionless \emph{i.e.} he distance between the ions is greater than the mean free path.
Due to the parker spiral, the magnetic field is perpendicular to the bulk velocity of the plasma, he solar wind is not, however, uniform.
The structure is divided into two mode, fast and slow solar wind. 
The difference between the two mode was highlighted by the readinds taken by the Ulessyes mission, while in a slingshot polar orbit around Jupiter and the Sun.
The SWOOPS instument on-board, measured the velocities of ions in the solar wind and found that the distribution is non uniform.
Above the poles, the plasma reaches velocities up to $800$ kms{^[]-1]} whereas, around the quiet Sun regions we find a lower range, $300$ - $400$ kms{^[]-1]}.

With such an energetic feature extending far into the solar system, how this interacts with the planets of the utmost importance.
Planets such as Mercury and Mars with little or no magnetic field to speak of, are bombarded by the energetic particles in the solar wind.
This is also the case for the moon, with small local magnetic fields (no global dynamo) it is directly exposed.
The moon provides the best opportunity to examine such a system, we therefore observe multiple different ways the solar wind interacts with it.
% citation of that lunar paper needed here
Primary among them is proton back scattering off the lunar surface as energetic neutral atoms, roughly $8$ - $28\%$ of the particles scattered are Hydrogen.
Conversely, we also observe sputtering, however He{^++} atoms are much better sputtering agents than H{^+}, so were we to mine the lunar surface, we might find significant resources of this element. 
As such the Suns relationship with these, generally, rockier, less well 'protected' planets and solar system bodies is always going to result in direct contact between the solar surface and the surface.

The alternative is clearly planetary bodies which produce a global magnetic field.
In the case of the Earth, the magnetic field dynamo is the convecting cells of the molten magma core, whereas, in the case of the gas giants, the extremely high density causes the hydrogen to be in a metallic form and convecting cells of hydrogen are believed to be the initiator of Jupiters magnetic field.

In this case we observe and extremely complex structure and system.
The solar wind approaches with the magnetic field perpendicular to the orbital plane and the Earths' magnetic field in an approximate dipole state.
Therefore, we have a case where two vertical magnetic fields meet.
This leads to an increase in the magnetic field strength, Ion density and potential, this feature is known as the bow shock.
All the elements of the solar wind that interact with the bow shock will be affected in some way.
Ions and particles backstream off the bow shock, reflected by the potential barrier formed by the increase in magnetic field strength, and the solar wind magnetic field wraps around that of the Earth.
Behind the bow shock we find the magneto sheath, a magnetically turbulent region comprising the 
material that has managed to pass through the bow shock.
The density of particles and the magnetic pressure decreases over the magnetosheath until the pressure from this region is balanced by the Earths magnetic pressure, whereby, the magnetopause is formed.

Given that this region of the Earths magnetosphere is heavily influenced by the solar wind, the exact structure of it is defined by the state of the solar wind.
As such, the fast solar wind will compress this region further as the magnetic pressure will be higher and events such as CME's which increase the particle density, also altering this region significantly.

As the solar wind magnetic field wraps around this region, it will eventually come into contact with the open magnetic field lines at the Earths poles.
This leads to a reconnection event, releasing the magnetic tension in the open fields lines and causing them to be dragged out behind the Earth, forming the magneto tail, and eventually reconnecting with magnetic field which has made the same journey on the opposite side of the Earth. 
This forms a current sheet roughly in the equatorial plane, however, this will be impacted and changed by the solar wind and geomagnetic events in the same way as the magnetosheath.

A second result of the open magnetic field lines at the Earths poles, is that particles from the solar wind may interact with, and consequently spiral down these open magnetic field lines.
Particles spiralling down a magnetic field line are accelerating and, therefore, have excess energy which is dissipated as a electromagnetic radiation.
This manifests as the phenomenon known as the Aurora, observed at high magnitude latitudes, appearing as a ring of light when viewed from space.
Hence, when the earth is bombarded by geomagnetic storms created by flaring and CME type events, we see warping of the bowshock/magnetopause system and increased appearance of Aurora over a wider range of latitudes.

The question becomes, is the magnetospheric model we observe at Earth, applicable to the gas giants?
The answer appears to be yes, firstly, we observe Aurora on the poles of both Jupiter and Saturn, strongly implying a similar dipole magnetospheric structure with open field lines over the poles.
Differentiating the two systems is the massive size of Jupiter's magnetic field, $20,000$ times larger than that of Earths and extending 100 times further.
This significantly larger extent is, in part, due to the reduced pressure from the solar wind, allowing the magnetosphere to expand more significantly, but also attributed to the more powerful dynamo (sufficiently strong, that the Van Allen belts that are toroidal at Earth are flattened out by Jupiters magnetic field).

The Suns influence ends with the termination shock.
At this point the outward solar wind pressure is finally in balance with the pressure of the interstellar medium.
The result is that the solar wind suddenly decelerates, causing a shock to form, it has been proposed recently that voyager 2 has made it across this boundary becoming the first man made object to leave the solar system.
The region between the solar system and the inter stellar medium, draws parallels with that between the solar wind and our magnetosphere.
There is a bow shock due to the Suns progress around the galactic disk, a heliosheath and heliopause, all of which have proxies in the Sun-Earth interaction.

For all of these reasons, it is clear that the study of the Sun, its local environment and the explosive events are essential to our continued existence.
Geomagnetic storms have previously knocked out power lines and will affect the operations of satellites in orbit, consequently, predicting and understanding these explosive events is essential.

\subsection{History of observations (generally)}

Mankind has been fascinated with the Sun for as long as we have been sentient, which is understandable given that it dictated seasons, whether crops grew and was a giant burning orb in the sky.
As such, in early civilisations it was regularly worshipped as a god, early Mesopotamian cultures worshipped Shamash, Rah was the Sun god of the Egyptians and Amaterasu is the Shinto godess.
Around these gods, mythologies were built; the Egyptian solar chariot, carryin the Sun across the sky, battling demons overnight to rise again on the other side; In Hinduism, the Sun god is named Surya and is driven across the sky in a seven horsed chariot representing the days of the week.
These mythos speak to the inherent importance of the Sun to civilisation on Earth, however it was not until the $17^{th}$ century that scientific observations really begin to occur seriously.


Fresh advancements in glass work and lens technology allowed the development of more sophisticated telescopes.
Utilising the telescope to project the image (as not to look directly at the Sun) observers hand drew their direct observations of the solar surface.
This was clearly the photosphere, due to its emissions being in the visible spectrum. 
The most prominent feature of the photosphere are the Sunspots appearing at the surface, therefore, the early telescopes allowed significant study of this feature.

\begin{figure}
	\includegraphics{'Figs/galileo_sunspot.gif'}
	\label{fig:gali_sp}
\end{figure}

Due to the nature of the observations, Galileo and his contemporaries records do not have the Suns equator normalised to the centre. 

The work of Johannes Kepler demonstrated that the Sun rotated on its axis and Sunspots were first observed by Harriot and the Fabricus family.

Initially there was debate as to what these dark features were, one hypothesis is that these were shadows of planetary bodies inside Mercurys orbit,  



Ground Based

Space based

Radio Observations


\section{Plasma behaviour}

\subsection{MHD}

\subsection{Reconnection}

\section{Jet and Macrospicules}

\subsection{The ejecta zoo}

\subsection{Macrospicules}

\subsection{Numerical Simulations}
































\begin{pycode}[chapter1]
from __future__ import print_function

ch1 = texfigure.Manager(pytex, number=1, base_path='./Chapter1/')
\end{pycode}

\begin{pycode}[chapter1]
fig = plt.figure(dpi=100, figsize=texfigure.figsize(pytex, height_ratio=1.5))

xx = np.linspace(0, 6*np.pi, 1000)
plt.plot(xx, np.sin(xx))

fig.tight_layout()
fig.subplots_adjust(bottom=0.15)
photosphere = ch1.save_figure('sin', fig, fext='.pgf')
photosphere.caption = "Some sine wave."
\end{pycode}

\py[chapter1]|photosphere|



Hello world, checkout \cref{fig:sin}.


\begin{pycode}[chapter1]
	
multi = texfigure.MultiFigure(4,1)
width = 0.9

X = [[5,6], [7,8], [5,6], [7,8]]
Y = [[1,2], [3,4], [1,2], [3,4]]

for x,y in zip(X,Y):
	fig = plt.figure(figsize=texfigure.figsize(pytex, scale=width))
	plt.plot(x, y, 'o')
	
	Fig1 = ch1.save_figure('test', fig)
	Fig1.subfig_width=r"{}\columnwidth".format(width)
	multi.append(Fig1)
\end{pycode}

\py[chapter1]|multi[0:2]|
\py[chapter1]|multi[2:4]|