% !TeX root = ../thesis.tex
%*****************************************************************************************
%*********************************** First Chapter ***************************************
%*****************************************************************************************

\newcommand{\pd}[2]{\frac{\partial #1}{\partial #2 }}
\newcommand{\td}[2]{\frac{d #1}{d #2 }}
\newcommand{\mb}[1]{\mathbf{#1}}
\newcommand{\divv}[1]{\bigtriangledown{#1}}
\newcommand{\del}{\bigtriangledown}

\label{ch:Intro}
\chapter{Introduction}  %Title of the First Chapter


%%%%%%%%%%%%%%%%%%%%%%%%%%%%%%%%%%%%%%%%%%%%%%%%%%%%%%%%%%% PAPER TEXT %%%%%%%%%%%%%%%%%%%%%%%%%%%%%%%%%%%%%%%%%%%%%%%%%%%%%%%%%%%%%%%%%%%%%%%%%%%%%%



\section{The Sun}
The Sun is an extremely complex system, consisting primarily of ionised plasma, the formation of which generated the collection of planets and asteroids we call the solar system.
As such, the study of our nearest star should be at the forefront of our research into the cosmos; any model we build to examine other stars must first accurately describe our star. 
The Sun takes its place on the Hertzsprung-Russell [\cite{Hertzsprung1909, Russell1914}] diagram as an early life main sequence star at the yellow end of the stellar spectrum.
It is a population 3 star, meaning that it has a high metallic content, a fact which aids in our observations immeasurably.

\subsection{From core to photosphere}

The structure of the Sun can be divided into internal and external regions.
Internal structure has been inferred by the use of techniques such as Helio seismology, therefore we still have many questions as to the exact mechanisms dominating below the photosphere.
At the centre is the solar core, wherein fusion takes place. 
Followed by the radiative zone, tachocline and convective zone.

The tachocline has the greatest impact on the solar atmosphere, as it has also been proposed as the source for a dynamo generating the magnetic field.
The most important result of the tachocline is that at this point, the motion changes from uniform behaviour of the radiation zone to the differential rotation of the convective zone.
This differential rotation is the cause of extremely complex global magnetic field permeating the Sun.
The tachocline, as discussed in \cite{Brun2001}, is a thin layer between the radiative zone and the convection zone, of which not a great deal is known.
A particular impact of the tachocline's existence, is that the $p-mode$ oscillations used in helioseismology, interact strongly with it, as demonstrated in \cite{Chaplin2014}, to the point of filtering high order oscillations out.
There have been recent studies, such as \cite{Obridko2007} which suggest that the tacholine is responsible for several $1.3$ to $3$ year cycles in feature observed higher in the solar atmosphere, however this has not been definitively proven.

Above the tachocline, there is the onset of convection, naming this region of the solar interior.
The convective cells of heating and cooling plasma carries the magnetic field from the tachocline up to the bottom of the photosphere.
The tops of these cooling cells, hotter plasma in the centre and cooler, darker, plasma at the edges form the granulation ubiquitous in any image of the photosphere [\cite{Nordlund2009}].


\subsection{The solar atmosphere}
The photosphere is the layer of the Sun which we can observe with the naked eye when used in conjunction with appropriate instrumentation, such as a camera obscura.
The photosphere is so called, because this layer emits light in the visible spectrum, and has a temperature range of $6000$ K at the base, and $4700$ K at the top. 
The granulation resulting from the convective region, \cref{fig:granssp}, appears as dark lines and bright 'bubbles' expanding and collapsing as hot material rises, cools and consequently falls back down into the solar interior, [\cite{Nordlund2009}].
These features tend to measure approximately $2$ Mm from one boundary to the other, which, are know as inter-granular lanes.

\begin{figure}
	\centering
	\includegraphics[width=0.75\textwidth]{Chapter1/Figs/granulation_and_sunspot}
	\caption{An image of the photosphere featuring its most prominent structures. The sunspot is comprised of the Umbra, dark plasma at the centre, and Penumbra, the thin structures leading away from the Umbra. It is through the Umbra that the magnetic field emerges, causing a lowering of temperature, hence its dark appearance.
	The rest of the image is dominated by granulation. See the tops of the cells as bright plasma surrounded by the darker, cooler, intergranular lanes
		\url{https://www.aps.org/units/dfd/pressroom/papers/sun.cfm}}
	\label{fig:granssp}
\end{figure}

The most prominent feature in the photosphere are sunspots.
These features are significantly cooler that the surrounding photosphere, with a dark centre, known as the umbra, through which open magnetic field emerges from the solar interior.
Spreading away from the umbra is the penumbra, a ring of long thin structures spreading away from the umbra.
These have been widely studied and have become known as fibrils.
The reason these are so widely studied, is that it has become clear that waves travel up these features from the solar interior.
Sunspots are normally part of an extremely complex system of magnetic field, meaning that there are usually many in one active region.
Sunspots are widely used as indicators of solar activity, the study of them has lead to the discovery of the 11 and 22 year solar cycles.
Analysis of their locations has also lead to the proposition of an active longitude, where sunspot formation is more frequent, the testing of this concept with macrospicules is the subject of \chapter{ch:4}.
%TODO citation for solar cycles discovery
The photosphere is the coolest point in the solar temperature profile, from this point the temperature of the solar atmosphere will continue to increase with distance from the photosphere.

The layer above the photosphere, is the chromosphere, so called, as when observed during solar eclipses, it appears colourful.
The density drops sharply from the photosphere to the order of $10^{-4}$ and is approximately $2$ Mm thick. 
Over this $2$ Mm layer the temperature rises from $4000$ K to $25,000$ K, see \emph{e.g.} \cite{Withrobe1977}.
This result is currently one of the primary focuses of the solar research community.
The chromosphere is incredibly complex in terms of its magnetic structure and the transport of heat, consequently the chromosphere plays an essential role in the formation of explosive events such as solar flares and coronal mass ejections (CMEs).

It is in the chromosphere, and subsequently the corona, where we now see the magnetic fields which are rooted in the sunspots we observe in the photosphere, demonstrated in \cite{Athay1976}.
These appear as large-scale loops of plasma which has been locked into the magnetic field lines, extending through this region and right into the corona, they regularly reach $10$'s of Mm into the atmosphere.
Lower in the chromosphere, we find a second, smaller-scale population of loops, and are a few Mm across on average.
Small-scale chromospheric loops have been presented as a possible link between the photosphere and the chromosphere, due the their most likely formation being a small-scale flux emergence event [\cite{Ulmschneider1982}].

%TODO possibly move elements to later on
%Examining any image of the chromosphere, it is almost impossible to miss the fabled `forest of spicules'.
%Spicules are small, short-lived, explosive events with origins in the chromosphere which extend through the transition region and into the corona, \cite{Beckers1972}.
%They are ubiquitous in the chromosphere, their number density being of the order of $10^5$ at any given time, however, their spatial distribution is not uniform.
%They are found to form on intergranular lanes, as such their is much debate about how they form.
%Candidates for the formation mechanism are; magnetic reconnection, \emph{p}-mode driven and plasma drains, although there is no conclusive evidence for any of these drives.
%Spicules high number density has lead many to propose them as a possible solution to the coronal heating problem, be that through instability or propagation of waves into the atmosphere directly.
%Recent in depth study has revealed the possibility of two populations of spicules, dubbed, Type-I and Type-II, a concept presented by \cite{DePontieu2007}.
%Type-I have been defined to be longer lived and less explosive, they are also observed to emerge and have a ballistic motion away from and returning to the Sun.
%However, Type-II show significantly more explosive velocities, up to $150$ km/s recorded during their initial formation, however, they are not observed to return and lifetimes are not expected to exceed $5$ mins.
%However, scientific consensus has yet to be reached, \cite{Zhang2012} do not find any separation in population, more on which in \cref{ch:2}.

Particularly noticeable in the chromosphere are the footprints of the coronal holes, a review of which can be found in \cite{Cranmer2009}.
These appear as regions of dark amongst the bright chromospheric features, this is a result of the cool plasma, lower in the atmosphere becoming visible through these holes.
During the minimum phase of the solar cycle, there are usually two prominent coronal holes at the solar poles. 
These can cover half of the solar disk during particularly inactive solar minima, however, at solar maxima, these polar coronal holes disappear as the magnetic field becomes increasingly complex.
The coronal holes are characterised by magnetic field lines extending up though the solar atmosphere, whereas, in the quiet Sun (areas not coronal holes) the magnetic field lines are closed, generally forming small and large scale loops.

A demonstration of this complex magnetic field, is the network bright point \cite{Muller1992}.
They are caused by the squeezing of granules and resulting flux compression.
Theses are prevalent throughout the solar atmosphere, we observe them in the photosphere and corona, however, they are primarily chromospheric features.
They are thought to be regions of very high magnetic pressure, causing lower gas pressure and therefore an increase in temperature.

There are several large-scale features with their roots in the photosphere.
Filaments, reviewed in \cite{Engvold2004}, and their off limb counterparts prominences, stand out spectacularly when observing in H$\alpha$.
Filaments are observed as dark strands of plasma, having risen from the cool photosphere against the hotter chromosphere, that can extend far across the solar disk and have lifetimes of many months.
Prominences are the same features observed over the limb, and hence appear as bright features against the dark sky.
This allows us to observe the impact on the atmosphere more closely. 
\cite{Kawaguchi1970} observe the plasma falling away from the main body of the prominence, becoming known as coronal rain.
The effect this rain has on the atmosphere is not yet fully understood, but there has be discussion as to its merits in terms of possible heating or cooling effects.

At the top of the chromosphere, the temperature of the plasma increases rapidly over a very short distance, approximately $500$ km. 
This is known at the transition region, \cite{Mariska1986}.
It acts as a barrier, and amplifier, between the chromosphere and corona with the temperature rising from $25,000$ K to $1$ MK. The mechanism which causes this is still not understood and is one of the prominent problems in solar physics.


Above the transition region we find the corona, \cite{Golub2009}, high temperature plasma whose behaviour is now dominated by the magnetic field which has emerged from the solar interior.
Iron is plentiful, which, we can tell is ionised from observations, and such the temperature has a minimum value of $1 \times 10^6$ K.
The corona is a complex system where exceptionally large scale features can have far reaching effects on the solar environment, [\cite{Reale2014}].

Here, we observe very large scale structures such as coronal loops and streamers, as well as transient explosive events such as solar flares and CME's, regularly in the same event.
Coronal loops, CME's and solar flares are tightly related.
As the footpoints of the coronal loops, sunspots rotate and migrate towards the equator. 
The magnetic tension of the loop increases to the point at which a reconnection event is the only way of reducing the magnetic energy in the system.
Solar flares are caused by large-scale magnetic reconnection events.
The material in the overlying loops becomes unbound as a result of the release of magnetic tension, and is released in the form of a CME, although this is not always the case.
The material ejected both in solar flares and CME's will then either fall back to the solar surface as coronal rain , or, will integrate into the solar wind.

The point at which the coronas bulk motion becomes radial, it is defined to be the solar wind.
At this point the plasma is dominated by the magnetic field and may be considered collisionless \emph{i.e.} the distance between the ions is greater than the mean free path.
Due to the Parker spiral, \cite{Forsyth2013}, the magnetic field is perpendicular to the bulk velocity of the plasma, the solar wind is not, however, uniform.
The structure is divided into two modes, fast and slow solar wind. 
The difference between the two modes was highlighted by the readings taken by the Ulysses mission, while in a slingshot polar orbit around Jupiter and the Sun.
The SWOOPS instument on-board, measured the velocities of ions in the solar wind and found that the distribution is non uniform.
Above the poles, the plasma reaches velocities up to $800$ km s${^{-1}}$ whereas, around the quiet Sun regions we find a lower range, $300$ - $400$ km s${^{-1}}$ [\cite{McComas2003}].

The Sun's influence ends with the termination shock.
At this point the outward solar wind pressure is finally in balance with the pressure of the interstellar medium.
The result is that the solar wind suddenly decelerates, causing a shock to form, it has been proposed recently that Voyager 2 has made it across this boundary becoming the first man made object to leave the solar system.
The region between the solar system and the inter stellar medium draws parallels with that between the solar wind and our magnetosphere.
There is a bow shock due to the Sun's progress around the galactic disk, a heliosheath and heliopause, all of which have proxies in the Sun-Earth interaction.

For all of these reasons, it is clear that the study of the Sun, its local environment and the explosive events are essential to our continued existence.
Geomagnetic storms have previously knocked out power lines and will affect the operations of satellites in orbit, consequently, predicting and understanding these explosive events is essential.

\subsection{Observations}

\subsubsection{History}

Study of the Sun took a large step forward when observatories such as the Royal Greenwich Observatory began taking measurements in 1874.
Now with consistent imaging from the same source and a history of smaller observations combined, larger patterns within sunspots was revealed.
Taking a monthly average of the sunspot count revealed a rise and fall in the sunspot number count over an $11$ period, now referred to as the solar $11$-year cycle.
Sunspot numbers in modern times are calculated by the number of sunspot groups multiplied by $10$ as that is the average number of sunspots in a sunspot group.
This definition is utilised by the National Oceanic and Atmospheric Administration (NOAA) in America and Solar Influences Data Analysis Centre in Belgium (SIDC) in Belgium.
Both of these organisations monitor the Sun and its impact on Earth including radio flux and total solar irradiance.
All of these can be used as a proxy to demonstrate the $11$-year solar cycle.

If we plot the date of the sunspots occurrence with respect to their latitude, we produce another diagram demonstrating the $11$-year cycle.
This is the now famous butterfly diagram \ref{fig:sunspot_count}, in which populations of sunspots tend to form closer and closer to the equator before a break point at which they begin forming further away, the time scale of which is $11$ years.

This is a result of the previously discussed differential rotation mechanism in action in the Sun's motion.
As the magnetic field becomes increasingly complex and the bands of magnetic polarity grow closer, increasingly more sunspots are pushed to the solar equator.
This demonstrates that the $11$-year cycle is result of the magnetic field and dynamo.
The Debrecen Photoheliographic Data (DPD) sunspot catalogue continued the work of the Greenwich catalogue, recording sunspot groups location and size.
This catalogue is the basis for the analysis of the macrospicules with respect to the Carrington rotation.

\begin{figure}
	\centering
	\includegraphics[width=\linewidth]{Chapter1/Figs/bfly}
	\caption{Plotting the latitude of the Sunspot against the time of its appearence is plotted above. It reveals the now characteristic butterfly shape as the Sunspots migrate to the equator over the solar cycle, as shown on the bottom. The area of sunspots as a percentage of the solar surface against the time at which it formed. It conveniently demonstrates the $11$ year solar cycle. 
	\url{http://solarscience.msfc.nasa.gov/SunspotCycle.shtml}}
	\label{fig:sunspot_count}
\end{figure}

\subsubsection{Observational Solar Spectroscopy}

Newton was the first to observe that light from the Sun could be divided into its component parts using a simple prism, but it was a very long time before we would reach a point where this information could be used for science.
Spectroscopy now forms the basis for almost all solar observations, the numerous heavy elements present in the Sun's atmosphere mean that it is extremely effective over range of temperatures.

\begin{figure}
	\centering
	\includegraphics[width=\linewidth]{Chapter1/Figs/sun_spectrum}
	\includegraphics[width=\linewidth]{Chapter1/Figs/fraunhofer_lines}
	\label{fig:fraunhofer}
	\caption{The top figure is a visible light measurement of the Sun. It demonstrates that there are elements in the atmosphere absorb the light and appears on this spectrum as a dark line. The Fraunhofer lines (absorption lines) are marked on a continuous spectrum with an intensity profile below.
		\url{http://media.radiosai.org/journals/Vol_05/01JAN07/04-musings.htm}}
\end{figure}


Figure~\ref{fig:fraunhofer} demonstrates the white light solar spectrum, immediately apparent are the dark absorption lines.
They are a result of the quantum mechanical effects of the electron energy shells around atoms and molecules.
In the case of the Sun, a continuous spectrum is radiated from the photosphere and this light then interacts with the elements higher up in the atmosphere.
Upon collision with an atom or molecule, the exact wavelength of light which corresponds to the energy required to excite an electron from one energy level to another is absorbed by that electron.
The direct result of this is that the wavelength absorbed in the energy transaction is absent from the white-light spectrum being emitted from the photosphere, and hence appears as a dark line when observed from beyond the solar atmosphere.
These transition lines are closely aligned to the temperature at which they are formed, therefore the higher emissions lines coincide with higher temperatures.

In the case of Hydrogen, the most prevalent atom in the Sun, the array of lines generated by the transitions between energy shells is known as the Balmer series, demonstrated in Figure~\ref{fig:balmer}.
The primary line in this range is H$\alpha$, which, is widely used in solar observations as it covers the photosphere and lower chromosphere.
The complexity is that the line is broad, for example, the H$\alpha$ CRISP filter at the SST (see \ref{sec:groud-based}) can measure $\pm 2$ nm about the main emission line $656.28$ nm.
This allows complex analysis of solar features, but the data must be handled carefully, environmental factors such as magnetic field and temperature can alter the emission line.
\ref{ch:5} utilises the extremely detailed images provided by CRISP to calculate dopplergrams of a jet-like feature at the solar limb.

\begin{figure}
	\centering
	\includegraphics[scale=0.5]{Chapter1/Figs/Balmer_series}
	\caption{A demonstration of the Balmer lines, transitions in the hydrogen atoms. In this case, specifically transitions between other lines and $n = 2$.
		\url{http://www.daviddarling.info/encyclopedia/B/Balmer_series.html}}
	\label{fig:balmer}
\end{figure}

% Maybe a quick paragraph on what lines are found where.

Above the transition region iron is bounteous, and so we have many emissions lines with which to examine the corona due to the high number of available transitions.
Fe VIII ($13.1$ nm), Fe IX ($17.1$ nm), Fe XII ($19.3$ nm), Fe XIV ($21.1$ nm), Fe XVI ($33.5$ nm) and Fe XVIII ($9.4$ nm) are all observed by the spacecraft Solar Dynamic Observatory (SDO) with the Atmospheric Imaging Assembly (AIA), \cite{Schmelz2013}.
As the temperature of the atmosphere increases, the amount of available energy changes and different transitions are excited.
The various lines here therefore apply to different temperatures.

% There are an array of other techniques (go into them a tiny bit)


\begin{figure}
	\centering
	\includegraphics[scale=3.5]{Chapter1/Figs/stokes_params}
	
	\caption{The variations of the polarisation of light by the magnetic field. Q-left to right and up and down. U, Inclined to $45$, and V, cicularly polarised.  
		\url{https://spie.org/publications/fg05_p12-14_stokes_polarization_parameters. A Field guide to Polarisation.}}
	\label{fig:stokes}
\end{figure}

A more suitable method by which we analyse the solar magnetic field is Stokes Polarisation Parameters, which describe the degree to which passing through a magnetic field has affected the electromagnetic radiation.
With this method, the total intensity of an optical beam is defined to be a sum of three forms of polarisation.
These three polarisations and the total are referred to as the polarisation parameters, I, Q, U and V, \cref{fig:stokes}.
I is raw intensity of the electromagnetic radiation, Q is the linear polarisation horizontally and vertically, U is also linearly polarized however is rotated $\pm45^\circ$ from Q and lastly V, which is polarised circularly, both left and right handed.
Using a combination of the 4 intensities it is possible to construct a vectorgram of the magnetic field.
This is the method used by the SDO instrument Helioseismic and Magnetic Imager (HMI) to examine features on disk.
These magnetograms, \cref{fig:mag_network} are ineffective at the limb, which is a limitation of the method.
This is because a feature at the limb has no background light source to use as an origin to test the result of any polarisation. 

The vector magnetograms are used widely when studying active regions, due to the very high magnetic field strength.
They can track magnetic flux cancellation which regularly leads to the formation of other features, such as solar flares and CMEs, due to the release of magnetic energy into the corona \cite{Welsch2006}.

We also observe the small magnetic bright point and network bright points, observed in the photosphere and chromosphere, respectively, \cite{SanchesAlmeida2010}.
They are an inherent part of what is known as the magnetic network which was visible in original measurements of the solar magnetic field over the boundaries of supergranular cells.
The magnetic field emerges from the supergranule boundaries as open magnetic field lines, extending up, into the chromosphere and above \citep{Hasan2005}.
Of course, the structure therefore takes roughly the same form as the photospheric observations in G-Band (a Fraunhofer emission line; $43.079$ nm), but with bright structures highlighting the supergranular lanes \ref{fig:mag_network}.
As a result, we have open magnetic field lines emerging with a canopy of closed magnetic field between the two boundaries.
A structure such as this can be a catalyst for physical processes such as reconnection.

It is within these regions that the magnetic bright points reside.
They appear as points of extreme intensities against the cool dark plasma of the interganular lanes, due to their intense $0.1$ T magnetic field strengths.
They are demonstrated to drift and move as the granules evolve \citep{Chitta2012} which has led the community to suggest that these points could diffuse large amounts of heat into the solar atmosphere.

\begin{figure}
	\includegraphics[width=\linewidth]{Chapter1/Figs/magnetic_network}
	\caption{Calcium K filter image of the low solar photosphere. Observer the regions of bright lines highlighting the intergranular lanes where the bright points appear. Clusters of intensity designate active regions where there is significantly more magnetic activity. 
		\url{http://science.nasa.gov/science-news/science-at-nasa/2008/02oct_oblatesun/}}
	\label{fig:mag_network}
\end{figure}

\subsection{Observational Platforms}

In the current solar observation climate, we have a wealth of information, coming from many different sources.
Understanding the functionality of these instruments in terms of the raw method is essential to rigorous science and accurate readings.

\subsubsection{Ground-Based Telescopes}
\label{sec:groud-based}
Solar physics spent its early development being applied in ground-based solar telescopes, as has already been discussed, particularly in centres such as Royal Greenwich Observatory and Meudon Observatory in Paris.
In the modern era, the most powerful instruments, delivering high cadence and spatial resolution, are based on the ground.
Due to the challenges of observing on the ground, the location of these facilities is carefully chosen, regularly at high altitude, minimising disturbance of the signal by the atmosphere, and away from busy, polluting cities for the same reason.
The main advantages of ground based telescopes over space bourne, is that the instruments can be heavier and literally more extensive given that they do not need to be transported into space.

On La Palma in the Canary Islands, at $2360$ m altitude, the Swedish Solar Telescope is situated.
It utilises a $1.0$ m mirror in a refracting, vacuum solar telescope.
The vacuum is necessary due to the quantity of light the mirror reflects, causing heating which would disrupt the signal being transmitted down to the receiver.
The 'receiver' is what has become known as the optical bench; here the light beam is sent to one of several possible processing suites which are easily accessible to structure observations as needed.

The SST has two such pipelines, one for the red end of the electromagnetic spectrum and another examining the blue end.
The CRisp Imaging SpectroPolarimeter (CRISP), focuses on the red end of the visible spectrum, whereas the soon to be updated instrument, Chromis, analyses the blue end.
The beam is split upon its arrival at the optical bench and sent to either of these instruments, but before the beam reaches CRISP there is a layer of correction known as adaptive optics.

Adaptive optics corrects for atmospheric scintillation, aberration and stabilises image motion.
These cumulative effects result in an optical path difference of the light incident with the lens/mirror of the telescope.
Given this information, it is crucial to know the atmospheric physics overlying and surrounding the telescope.
It is for this reason that observatories such as the Big Bear Solar Observatory \citep{Cao2010} in Los Angeles and the proposed new Chinese Giant Solar Telescopes \citep{Liu2014} are built on and by lakes, since the temperature will be lower, thus reducing heat haze, and therefore atmospheric effects.
The primary atmospheric parameters influencing the setup of the adaptive optics are the Fried parameter (which measures the quality of optical transmission), the Greenwood frequency (the bandwidth required for optimal correction) and the atmospheric turbulence profile (which allows a calculation of the refractive index of the atmosphere) as shown in \cite{Rimmele2011}.

As a result of the above points, the images from the SST are extremely customisable.
They are usually returned as data cubes, the dimensions of which are time, wavelength, x and y.
The resolution of these features are therefore changable dependant on the requirements of the observation.
Cadence can be as low as $2.5$ s, the spectral increment, $0.02$ nm, and spatial resolution of $0.12$ arcsec.
A part of what makes this form of observation viable is the ability to directly download the received signal, which is a limitation of space borne instrumentation.
Given the possibilities of ground based observations, the method is popular, with the aforementioned Big Bear Observatory joined by a plethora of other facilities, Richard B. Dunn solar telescope (DST) in New Mexico, Mauna Loa Solar Observatory on Hawii, which will shortly be joined by the Daniel K. Inouye Solar telescope (DKIST).
DKIST promises to be the most powerful solar telescope ever created with a $4$ m mirror, enabling resolution of $10$ km per pixel and overcoming the current quantity of photons issues at the limb for spectropolarimetry.



\subsubsection{Space Based Telescopes}

Space bourne telescopes have a significant advantage over their ground based counterparts due to the nearly constant un-interrupted, view of the Sun without the atmospheric effects disturbing the signal.
As a result, the volume of space based instrumentation has grown exponentially since the early Skylab missions taking solar images from a low Earth orbit.

The first solar mission to move on from imagers on space stations was the Solar Heliospheric Observatory (SoHO).
Placed at the gravitationally stable Lagrange point, L1, between the Sun and Earth, it affords constant viewing of the Sun.
It was one of the first missions to comprehensively cover the entire solar environment, the instruments of which are described in \cite{StCyr1995}.
The science covers dopplergrams of the photosphere, EUV imaging of the chromosphere and out to LASCO, monitoring the solar wind.

Following SoHO, the Advanced Composition Explorer (ACE), \cite{Garrard1997}, was placed near the L1 point as well.
It was launched in $1997$ with a much more specific goal of analysing the contents of the solar wind, extremely pertinent as the constituents and energy inherent in the solar wind can have extensive effects at Earth.
As a result of this, the scientific mission of ACE is heavily leant towards instrumentation which physically measures the energy and identity of a given particle.
The Solar Wind Ion Mass Spectrometer (SWIMS) and Solar Wind Ion Composition Spectrometer (SWICS), \cite{Gloeckler1992}, are used to measure these properties.
SWICS was initially built for the ULYSSES \cite{Ulysses1992} mission which orbited the Sun in a novel slingshot orbit around Jupiter, normal to the plane of orbit.

The results this instrument produced provide one of the most iconic sets of readings in the modern era of solar physics.
As Ulysses completed its orbit, the radial velocity profile of the solar wind was found to vary over solar latitude, \cref{fig:ulysses_sw}.
At lower latitudes, the solar wind was found to be of the order $200$-$400$ km s$^{-1}$, before getting to higher latitudes and a distinct transition to much higher velocities.
At the time of this first orbit, the Sun was at a solar minimum, consequently there were clear and distinct coronal holes and quiet Sun, matching with the fast and slow wind respectively.

\begin{figure}
	\includegraphics[scale=0.5]{Chapter1/Figs/ulysses_solar_wind}
	\caption{The landmark solar wind measurements. On the left the solar wind during minima. A strictly divided solar wind in which high velocities are detected over the poles and lower values over the more magnetically complex equatorial quiet Sun. On the right, almost homogeneous high velocities at the solar max. This is due to a greater amount of large scale structures, such as streamers and CME's, producing fast solar wind all over the solar atmosphere.
		\cite{McComas2003}}
	\label{fig:ulysses_sw}
\end{figure}

The confirmation of this association between wind mode and magnetic environment came in the next orbit of Ulysses.
In this orbit, the Sun was at maximum, therefore, it was almost entirely constituted of quiet Sun magnetic environments.
During this orbit, the solar wind was found to be uniformly chaotic , \ref{fig:ulysses_sw}.
Velocities were found to climb as high as those found in a fast solar wind mode, however, it is not the uniformly distributed fast mode emitted from the coronal holes, as evident in the solar minima readings.

Given that we now know that the solar wind and corona have significant reach and influence, a new mission was designed and initiated, the Transition Region and Coronal Explorer (TRACE) \cite{Gaeng1998}.
As the name suggests, this instrument was designed with the specific intention of investigating the higher reaches of the solar atmosphere.
Specifically, the aim of the mission was to investigate the three-dimensional structure of the low plasma beta atmosphere.
TRACE is a uniquely designed telescope, following the popular Cassegrain design, in which the primary mirror is divided up into 4 quadrants with separate coatings in order to filter the incoming light.
This method allows simple and efficient image co-aligning.
Given that TRACE was to observe the high atmosphere, Extreme Ultra Violet lines (with the rest of the EM spectrum filtered at entrance to the detector) were selected for the imaging instruments. 
These consisted of Fe IX $17.1$ nm, Fe XII $19.5$ nm and Fe XV $28.4$ nm.

The Japanese Aerospace Exploration Agency have organised several solar missions.
Solar-A, renamed Yohkoh, \cite{Tsuneta1991} upon its successful launch and commencement of observations, pre-dates SoHO.
It covered soft and hard X-Ray ranges and spectrometers covering, specifically, the coronal iron lines and a wide band spectrometer. 
Yohkoh was particularly successful with respect to the detection of high energy events producing large amounts of energy and X-ray emission, such as coronal jets and solar flares.
Building on the success of Yohkoh, a new mission was planned, sequentially named Solar-B.
It was later renamed Hinode, Sunrise in English, upon its successful launch and is presented in \cite{Kosugi2007}.
 
Hinode's launch in 2007 signified a significant move forward in space based solar observations, building from what was successful in Yohkoh.
The mission introduced small wavelength increment spectroscopy to space based missions.
The EUV Imaging Spectrometer (EIS) is designed to examine the chromosphere using two specific imaging techniques.
It is extremely flexible with 4 slit or slot positions, $1"$ pixel slit, $2"$ pixel slit, $40"$ pixel slot and $266"$ and two different modes for spectroscopy.

Possibly the most adventurous mission to probe the solar environment to date, is the STEREO mission (Solar TErrestrial RElations Observatory) described in \cite{Kaiser2008}.
As is suggested by the name of this mission, the primary focus was to form a more comprehensive picture of the solar atmosphere, by positioning two satellites in such an orientation to build a three dimensional picture.
Consequently, two identical satellites were launched into orbit, ahead of and behind the Earth.
There is an inherent differential in the angular velocity of the spacecraft, in order to move the spacecraft ever further around the orbit at approximately $45^\circ$ per year.
The primary objective of this stage of the mission was to obtain the optimal angle to produce three dimensional images of the Sun using a tomographic technique.

Given that STEREO was designed to give varying angles of the Sun-Earth  enviroment, the instruments on the mission are tailored to this need.
SECCHI is a suite of 5 imagers, utilising white-light coronographs, an extreme ultra violet imager and two wide angle Heliospheic Imagers (HI) designed to track CMEs to $1$ AU.
IMPACT detects solar wind electrons and the in-situ solar wind magnetic field strength and vector, while PLASTIC measures the composition of heavy ions, alpha particles and protons. 
Given the instumentation, STEREO is used extensively by space weather forecasters at NOAA, which will increasingly become an essential part of our lives.

STEREO's positioning is advantageous to this work as it affords a different viewing angle on features.
This can come in useful when considering features at the limb.
If STEREO is positioned correctly, it aid in removing uncertainty when considering possible line of sight effects.

Building on all of the above missions, the Solar Dynamic Observatory (SDO) mission began in 2010 \cite{Kaiser2008}.
The instruments are developments of concepts used on previous missions but expanding them to allow constant viewing of the entire solar disk.
Consequently all instruments on SDO view the full solar disk, at all times, maintaining the same temporal cadence. 
Therefore, the SDO mission produces significantly more raw data than any previous missions.
From a purely engineering standpoint, the instruments were carefully selected to facilitate the downloading of the data.
As such, SDO has three instruments; the Helioseismic and Magnetic Imager (HMI) examining the solar variability and finer scale structure of the solar magnetic field; the Extreme Ultraviolet Variability Experiment (EVE) measures the total solar irradiance in the Extreme Ultra Violet section of the spectrum and the Atmospheric Imaging Assembly (AIA) investigating the upper chromosphere and corona.

AIA's imaging suite provides full disk images in $4096 \times 4096$ resolution and most importantly at $12$ second cadence.
However, its distinguishing feature is the array of wavelengths analysing the atmosphere \cite{AIAspec} associated with various temperatures corresponding to the appropriate electron transitions.
The instrument ranges through $170$, $30.4$, $160$, $17.1$, $19.3$, $21.1$, $33.5$, $9.4$ and $13.1$ nm providing a temperature range of $5000$ K to $1.6 \times 10^7$ K.

Without SDO the work in this thesis would not be possible.
The constant full disk viewing greatly increases the chances of capturing the total evolution of a macrospicule.
As such, it is used extensively in all of these studies as the main observational tool in \cref{ch:3}, \cref{ch:4}, and plays an essential roll in \cref{ch:5} 


\section{Plasma behaviour}

Stars are incredibly complex features despite their apparent simplicity when viewed by the naked eye.
Their behaviour is entirely unlike any planetary body, and as has already been discussed, their inherent magnetic field makes their structure extremely complex.
All of these affects can be traced back to the fact that the Sun consists of gas kept at high temperature and pressure, which causes it to form the $4^{th}$ state of matter, plasma.
Plasma is defined as a gas in which the molecules reached an energy level that cause them to eject their outermost electrons and become ions, causing the gas to be a neutral mixture of charged ions and free electrons (produced from ionising the molecules).
It can be formed in several situations on Earth, such as a discharge of current from the atmosphere to the ground, manifesting as lightning as the propagating current ionises the air.

The motion and behaviour of a plasma can be defined on small scales by 3 factors; the plasma approximation, bulk interaction and the plasma frequency.
The plasma approximation states that the particles must be close enough together that any given particle must influence all particles within the Debye screening length.
The length is dependant on the permittivity of free space, the Boltzmann constant, electron charge, temperatures of the ions and electrons, density of electrons and density of an atomic species and can be written as.

\begin{equation}
	\lambda_D = \sqrt{(\epsilon_0\kappa_BT_e)/(n_eq_e^2)}
\end{equation}

Bulk interaction refers to the statement that the Debye screening length is small compared to the overall scale of the plasma, this implies that the motion of the interior guides the characteristic behaviour, rather than motion at the edges.
Lastly the plasma frequency refers to the oscillation of the elections within the plasma, which is valid in the case that the frequency is higher than the collisions between electrons and neutrals.
In this case, the electrostatic interactions dominate over the standard, gas-like behaviour we would otherwise see.
Of course, the case where all the molecules or atoms are ionised is the ideal case, however, in nature this is not always true.
The degree of the ionisation is defined in terms of the ratio of ions to electrons, $\alpha = n_i/(n_i + n_n)$ where $i$ is the number density of ions and $n$ for neutrals.

As is evident from the Debye screening length, the temperture of the plasma can have a dominating effect on the characteristics of the plasma.
The temperature is a measure of the kinetic energy of the plasma, clearly higher kinetic energy therefore equates to a higher temperature of the plasma.
However, electrons will reach a thermal equilibrium significantly faster than the ions and neutrals, in which situation the plasma will have two, or even three, populations.
In the case where the plasma's electrons and ions are in thermal equilibrium with the neutrals the plasma is said to be thermal.
In nonthermal plasmas, the electrons, whose temperatures raise quicker, will be at a higher temperature than the heavier ions and neutrals.
The degree of ionisation of the plasma that results in a thermal equilibrium is defined;

\begin{equation}
	x^2/{x - 1} = (2\pi m_e)^{3/2}/h^3 (\kappa_BT)^{5/2}/p_{gas} exp(-{\chi/\kappa_BT}), \cite{Saha1920}
\end{equation}

where $p_{gas}$ is the gas pressure, $m_e$ is the mass of the electron and $\chi$ is the ionisation energy (energy required to remove the least bound of the electrons around an atom).
This form of the Saha ionisation equation will hold for Hydrogen, but does not take into account multiple ionisation processes, as would be the case for a more complex atom or molecule.
The result of this is that degree of ionisation in a gas will increase with increases in temperature.
It therefore follows that not all plasmas are fully ionised.
At lower temperatures, the heavier elements won't gain enough energy to ionise, sometimes despite the fact that the electrons are orders of magnitude higher in temperature.

Plasma dynamics are significantly more complicated than those of a gas, due to the presence of inherent magnetic field. 
We therefore need a set of laws to define how the plasma behaves on scales such as those applicable on the Sun and in the atmosphere.


\subsection{Magnetohydrodynamics}

Magnetohydrodynamics are the set of laws by which we describe the motion of plasma on large scales.
They were derived by \cite{Alfven1942}, an achievement for which, Alfv{\'e}n was awarded the Nobel Prize.
The rules set up are a combination of the gas pressure equations and Maxwell's laws of electrodynamics, so let us now examine this relationship.

When considering a plasma, it is important to remember that while the total charge of the plasma will be quasi-neutral, the ions and electrons which constitute the mixture still carry charge.
Consequently, motions in the plasma will cause the charges to have a change in velocity.
In accordance with Faraday's law, a moving charge will cause a magnetic field to be induced and Ohms' law will also become a factor with charges moving through a magnetic field.
As such, this can be an extremely complex problem. 
Let us begin our discussion with Maxwell's equations;

\begin{align}
	\nabla \times \mb{E} &= -\pd{\mb{B}}{t}  &\quad \textnormal{Faradays Law}\\
	\nabla \times \mb{B} &= \mu_0\mb{j} + \frac{1}{c^2}\pd{\mb{E}}{t} &\quad \textnormal{Amp{\`e}re Law}\\
	\nabla\cdot\mb{E} &= \frac{\tau}{\epsilon_0} &\quad \textnormal{Gauss' Law}\\
	\nabla\cdot\mb{B} &= 0 & \quad \textnormal{Gauss' law of magnetism}\\
\end{align}

\noindent where $\mb{E}$ is the electric field stength, $\mb{B}$ the magnetic field, $t$ is time, $c^2 = (\epsilon_0\mu_0)^{-1}$,$\epsilon_0$ is the vacuum permittivity, $\mu_0$ is the permeability of free space, $\mb{j}$ is the current density and $\tau$ is the charge density.
Faraday's law describes how a changing magnetic field would induce an electric field, hence it is also known as the Induction equation \citep{Goedbloed2004}.
Amp{\`e}re's law describes the manner in which the magnetic field integrated around a closed loop, related to the electric current passing through said loop.
Gauss' law for magnetism is also known as the Solenoidal condition and states that no magnetic monopoles exist and the eponymous law describes the resulting electric field caused by an electric charge.

The second `half' of the magnetohydrodynamic equations are the laws for gas dynamics, expressed in terms of the partial derivatives:

\begin{align}
	\pd{\rho}{t} + \nabla \cdot (\rho\mb{v}) = 0 & \quad\textnormal{Equation of mass conservation}\\
	\pd{p}{t} + \mb{v}\cdot\nabla p + \gamma p\nabla\cdot\mb{v} = 0 & \quad \textnormal{Conservation of entropy}
\end{align}

\noindent in the Eulerian time-dervative reference frame, evaluated for a fixed position within the fluid.
In the case of these two sets of equations, there is currently no link other than the velocity vector, $\mb{v}(\mb{r},t)$ which can be introduced through the equation of motion for a fluid element.
The equation for the motion of the fluid element is derived from the rate of change of momentum equations 

\begin{equation}
	\frac{d}{dt} \int_{V}^{}dV\rho\mb{v} = \int_{V}^{}dV\rho\td{\mb{v}}{t} = \textnormal{rate of change of momentum}
\end{equation}

\noindent In this case the rate if change of momentum will equal the net force on the fluid element, in accordance with newtons second law, therefore;

\begin{equation}
	\rho \td{\mb{v}}{f} = \rho\mb{g} + \nabla\cdot[X]
\end{equation}

\noindent where X is the total of all forces exerted on the fluid element. 
Therefore the equation which will incorporate all of these terms and calculate the acceletation on a fluid element is:

\begin{equation}
	\rho \td{\mb{v}}{t} = \mb{F} \equiv -\nabla p + \rho\mb{g} + \mb{j} \times \mb{B + \tau\mb{E}} 
\end{equation}

\noindent As we are currently assuming that we have a totally ionised fluid, therefore the electric field is defined as

\begin{equation}
	\mb{E}' \equiv \mb{E} + \mb{v} \times \mb{B} = 0
\end{equation}

\noindent and therefore, $\mb{E}'$ in a co-moving frame will vanish.
The last assumption we need to make, is that the velocity of the plasma is not relativistic $v \ll c$.

This allows us to make some assertions as to the scale of the terms in Amp{\`e}re's equation, the length scales of $l_0$ and $t_0$ are shown such that $v = l_0/t_0$.
This means we can neglect the displacement current from Amp{\`e}re's Law and define the current $\mb{j}$ purely in terms of $\mb{B}$:

\begin{equation}
	\mb{j} = \frac{1}{\mu_0}\nabla \times \mb{B}
\end{equation}

As a result of which, we can neglect the effects of space charge on the plasma.
Additionally, the non-relativistic assumption means that we can also make a simplification to the acceleration of a fluid element equation as the electrostatic acceleration is small, as well, Gauss' law can be dropped through lack of need.
As such the electric field can be expressed merely in terms of the velocity and magnetic field vectors:

\begin{equation}
	\mb{E} = -\mb{v} \times \mb{B}
\end{equation}

But applying the above assumptions, and substituting in for $\mb{E}$ and $\mb{j}$, we obtain the basic equations of ideal magnetohydrodynamics (MHD).

\begin{align}
	\pd{\rho}{t} + \nabla\cdot(\rho\mb{v}) &= 0 \\
	\rho(\pd{\mb{v}}{t} + \mb{v}\cdot\nabla\mb{v}) + \nabla p - \rho\mb{g} - \frac{1}{\mu_0}(\del \times \mb{B}) \times \mb{B} &= 0\\
	\pd{p}{t} + \mb{v}\cdot\nabla p + \gamma p\nabla\cdot\mb{v} &= 0\\
	\pd{\mb{B}}{t} - \del \times (\mb{v} \times \mb{B}) &= 0\\
	\nabla\cdot\mb{B} &= 0
\end{align}

\noindent These equations are therefore applicable to the case where; 1) the plasma is strongly collisional, such that the time-scale of the collision between the particles is much smaller that the characteristic time scales of the entire system.
2) The resistivity of these collisions is small \emph{i.e.} the magnetic diffusion time scale much be longer than any other process occurring within the plasma.
3) The time-scale must be greater than that of the kinetic processes occurring within the plasma, such as ion gyration, Landau damping and length-scales longer than the ion skin depth and Larmor radius, \cite{Goedbloed2004}.

By making choices with respect to the units for length, mass and time, the MHD equations can be made dimensionless.
A typical length scale can be chosen such as $l_0$ to be something sensible and $\rho_0$ and $B_0$ are chosen from a representative point in the plasma and the time unit can be inferred from a basic speed of the plasma, \emph{e.g.} the sound speed or Alfv{\'e}n speed.

\begin{equation}
	v_0 \equiv v_{A,0} \equiv \frac{B_0}{\sqrt{\mu_0\rho_0}} \textnormal{which leads to} t_0 \equiv \frac{l_0}{v_0} 
\end{equation}

\noindent The density, velocity, magnetic field etc. are then used to define new dimensionless parameters and substituted back into the MHD equations, which remain unchanged but now a have an operator for these variables instead of the variable themselves.
The crucial outcome here is that the equations are not dependent on the size of the plasma evaluated, the magnetic field strength, the density or the time scale.
After scaling $l_0$, $B_0$ and $t_0$, the pressure term becomes of vital importance and is linked to the ratio between the kinetic pressure of the plasma and magnetic pressure.
This ratio is commonly referred to as the plasma beta, which is defined as:

\begin{equation}
	\beta \equiv \frac{2\mu_0p_0}{B_0^2}
\end{equation} 

\noindent This is and extremely useful flag when considering the behaviour of a plasma at a less precise level, as it indicates the forces dominant in a region.
If $\beta \gg 1$ the kinetic pressure terms are dominant, meaning that the kinetic motions of the plasma will determine its overall behaviour, such as in the photosphere and below.
Whereas, in the chromosphere and upwards, the balance more favours the magnetic field and the gas movement is determined by magnetic effects.
This leads us to another important result of ideal MHD, the frozen in condition.

We must first discuss the magnetic Reynolds number, which gives a dimensionless value to the ratio between the induction and diffusion taking place within the plasma.
It can be defined, therefore, as a ratio of the two.

\begin{equation}
	\Re_m = z\frac{VB/L}{\lambda B/L^2} = \frac{LV}{\lambda} 
\end{equation}

\noindent where $B$ is the typical magnetic field, $V$ is the typical velocity, $L$ is the typical length and $\lambda = c^2/{4\pi\sigma}$ is the magnetic diffusivity \citep{Choudhuri1998}.
We can conclude that the Reynolds number depends significantly on the overall size of the plasma, and is therefore almost always large for astrophysical plasmas.
We can now apply this to the induction equation in the context of said large scale astrophysical plasma.
Here, we will examine the ideal MHD limit, which states that the plasma will be infinitely conductive. 
We can therefore write;

\begin{equation}
	\frac{\mb{B}}{t} = \del \times (\mb{v} \times \mb{B})
\end{equation}

\noindent Then using the identity $\frac{\mb{Q}}{t} = \del \times (\mb{v} \times \mb{Q})$, which is applicable to the above equation and results in;

\begin{equation}
	\pd{}{t} \int_(S)^{} \mb{B} \cdot d\mb{S} = 0
\end{equation}

\noindent we discuss this equation in terms of the Lagrangian operator, as we wish to follow integrated surface, as applied to the plasma, as a fluid element and examine the variations in time.
We can therefore deduce that the plasma will move with the magnetic field lines, this is known as Alfv{\'e}ns theorem of flux freezing.
Take a large-scale plasma such as those that originate in solar active regions. 
When a reconnection event occurs, magnetic fields can become detached from the solar atmosphere and can be ejected into the solar wind.
The frozen in condition implies that the plasma which was on those field lines, will be carried with it.
Alternatively, in a situation where the gas pressure is dominant low in the atmosphere, a gas perturbation will cause the magnetic field to wave as well.
 


\subsection{Reconnection}
\label{sec:recon}

The above processes are considered to be ideal cases for energy emission, for example, the dissipation of MHD waves is an ideal solution to the problem, converting magnetic energy into kinetic energy.
The process is referred to as ideal, as it fulfils the conditions we highlighted above, that the plasma obeys Alfv{\'e}n's theorem of frozen in-flux, and that 3 premises of effects of a kinetic scale are obeyed.
However, in the reconnection process the frozen in condition is violated, as two plasma blobs which were connected by a field line, will now no longer be.
It also, impinges on the assumption that the magnetic diffusion occurs on time scales significantly greater than the dynamic phenomena within the plasma, as reconnection takes place on much shorter time scales.
This means that the reconnection mechanism itself is non-ideal.
In such processes, the magnetic energy is converted to kinetic energy and heat, more on which later.

Reconnection was first proposed as a mechanism for magnetic X-type null points, such as those found in solar flares as demonstrated in  \cite{Giovanelli1946}.
The null point is considered to be the point at which the magnetic field vector has a resultant vector equal to zero, something which regularly occurs when two magnetic fields are interacting, for example during flux emergence.
Subsequently this was followed up by \cite{Dungey1953} who demonstrated that the collapse of the magnetic field near an X-type null point would lead to the formation of a current sheet.
He also proposed the concept of breaking the magnetic field lines.

The first MHD model to describe a reconnection event, was the Sweet-Parker model \citep{Sweet1958, Parker1957}, where the phrase, 'reconnection of field lines' was first utilised.
Although, the model was flawed, the reconnection it predicted was insufficiently quick to describe a solar flare and subsequent massive release of energy.
In the Sweet-Parker model, the current sheet that forms as a result of the interacting magnetic fields is on the global scale of the system, whereas \cite{Petschek1964}, utilised a current sheet many orders of magnitude smaller than that of Sweet-Parker.

In order for us to build a model for a reconnective event, we will need a few components, the magnetic field equations that describe the null point and a description of the current flowing over the change in magnetic field.
Let us begin our discussion with the null points, of which, there are two kinds, elliptical and `X-type'.
The `X-type', which can be applied to who seperate magnetic field systems coming together, is the situation which occurs most regularly within the solar atmosphere so we shall consider this mechanism.

In the situation, the field lines are hyperbolic, bending away from the centre, forming an X-type neutral point, of the type applied to the solar flare problem by the authors above.
The limiting field lines are defined in terms of their angle of separation.
These lines go through the origin, and are known as separatrices, and form the characteristic 'X-type' null point. 
The $\bar{\alpha}$ value, which defines the angle between said separatrices, is related to the current density.
Which we can calculate by taking the curl of the magnetic field, \cite{Priest2007}

\begin{equation}
	\mb{j} = -\frac{B_0}{\mu_0L_0}(1 - \bar{\alpha}^2)\hat{\mb{z}}
\end{equation}

\noindent Where $\hat{\mb{z}}$ points out of the $xy$ plane.
Now we have reached the point where the current density can be discussed with respect to the null point.

The current sheet typically appears at neutral points where there is a tangential discontinuity, in which case the the magnetic field is tangential and the plasma flow across the current sheet is zero.
When the system is in equilibrium, the plasma either side of the sheet, and in the sheet itself, are in pressure balance.
Usually, the total magnetic field on the current sheet will be zero, and we assume that the ambient pressures vanish, we can say that the magnetic pressure either side of the sheet is equal to the gas pressure on it:

\begin{equation}
	\frac{B_2^2}{2\mu} = p_c = \frac{B_1^2}{2\mu}
\end{equation}

\noindent where $B_1$ and $B_2$ are the magnetic fields either side of the current sheet and $p_c$ is the gas pressure at the current sheet.
It follows that the magnetic field undergoes an exact reversal in the magnetic field.
If it is the case that the magnetic field within the sheet is parallel to the $y$ axis and varies in $x$, is defined $\mb{B} = B_y(x)\hat{\mb{y}}$, applying Amp{`e}re's law can give us the current density in $z$:

\begin{equation}
	j_z = \frac{1}{\mu}\frac{dB_y}{dx}
\end{equation}

\noindent This means that when we get a steep gradient in $B_y$ with respect to $x$, a strong current along the current sheet is produced perpendicular to the field lines.
This is the current sheet that lies at the heart of reconnection models, however the tangential discontinuity is susceptible to instabilities.
Now that we have the null point between the two magnetic fields and the current sheet that forms as a result, we have the environment necessary to begin the reconnection of magnetic field lines.

Let us consider the induction equation again defining how the magnetic field changes with respect to the magnetic field,

\begin{equation}
	\pd{\mb{B}}{t} = \del \times (\mb{v} \times \mb{B}) + \eta\del^2\mb{B}
\end{equation} 

\noindent with the first term representing the advection and the second the diffusivity of the magnetic field.
The ratio of these two terms can be represented by the magnetic Reynolds number, $\Re_m = LV/\lambda$, and of course, this value will dictate the evolution of the induction.
The diffusion term dominates in the case where $\Re_m \ll 1$, however, we must be aware that this condition will not apply on large scales in the solar atmosphere, although it may be true on very small scales.
The magnetic field can be defined in terms of a convection equation, $\pd{B}{t} = \eta\frac{\partial^2B}{\partial x^2}$, describing the way the field diffuses away.
We are assuming a thin current sheet in this case, with the magnetic field diffusing at a rate of $\eta/l$, where $l$ is the width of the sheet and changes with $(\eta t)^{1/2}$, increasing with time.
Whereby the magnetic field at a fixed position decreases, it is concluded that it has been annihilated.

The second mechanism for the changing of the magnetic field lines is advection. 
Under the condition $\Re_m \gg 1$, we drop the diffusive term from the induction equation.
On these scales the frozen in condition applies, and the plasma is free to move up and down the magnetic field lines.
In this scenario we define the velocity field lines to trace out a hydrodynamic stagnation point flow.
This causes the field lines to flow inwards, and accumulate at $x = 0$ at which point the magnetic field increases significantly.
We also find that when the induction equation is considered in $y$, there is also a build up of field at $x = 0$, growing exponentially with time.

Examining the current sheet, and the accompanying increase in magnetic field stengths leads to a problem; on small scales of $X$, the diffusivity terms take over.
Therefore there must exist a balance at which the rate of plasma transporting magnetic field in, must be balanced by the rate of diffusion away from $x = 0$, producing a steady state.
Consequently, at $x = 0$ we have the two magnetic fields meeting and cancelling, and in the case where resistivity is small, the magnetic gradient will be large, causing increasing amounts of magnetic energy to be converted to kinetic.
While the magnetic field lines will disappear, the plasma which was formerly hosted on them will not, the result of this is that it will flow out of the system in $z$.

One of the first papers to demonstrate this in action at the solar surface was \cite{Innes1997}.
The authors use the SUMER instrument to examine the Si IV $139.3$ nm in two $90$ minute observing windows with rastering consisting of $5$ s exposures.
The authors captured a $4$ minute explosive event, $90$ s after the onset of the feature the doppler shifts maximised, and the event had doubled in size in the east-west direction.
They found that the asymmetries in the emission line, mirror those expected of a bidirectional jet, the flow axis of which is away from the line of sight and the centre of the two flows is paired with the origin of the jet.
The velocity is calculated from the doppler shift, and given as $100$ km s$^{-1}$.
The apparent length of the jet $0.4$ Mm, giving a true jet length of $1.2$-$2.4$ Mm, therefore the jet stretches across the chromosphere and possibly encroached on the corona.
This length falls in the lower end for macrospicules, however its lifetime is $3$ minutes, which is a little low for a macrospicule.

In \ref{ch:2} I will examine the current literature with respect to the plethora of solar jets, throughout the solar atmosphere.
\ref{ch:3} will present the a work examining the statistical properties of macrospicules.
\ref{ch:4} examines the locations at which macrospicules are generated over the Carrington rotation.
In \ref{ch:5} a case study is presented of a macrospicule observed using CRISP/SST and AIA/SDO.
Lastly I will make my concluding remarks, summarising macrospicules place in the zoo of solar ejecta.


