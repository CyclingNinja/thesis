% !TeX root = ../thesis.tex
%*****************************************************************************************
%*********************************** First Chapter ***************************************
%*****************************************************************************************
\label{ch:Intro}
\chapter{Introduction}  %Title of the First Chapter


%%%%%%%%%%%%%%%%%%%%%%%%%%%%%%%%%%%%%%%%%%%%%%%%%%%%%%%%%%% PAPER TEXT %%%%%%%%%%%%%%%%%%%%%%%%%%%%%%%%%%%%%%%%%%%%%%%%%%%%%%%%%%%%%%%%%%%%%%%%%%%%%%



\section{The Sun}
The Sun is an extremely complex ball of plasma, the formation of which generated the collection of planets and asteroids we call the solar system.
As such, the study of our nearest star should be at the forefront of our research into the cosmos, any model we build to examine other stars must first accurately describe our star. 
The Sun takes its place on the Hertzsprung-Russell diagram as an early life main sequence star at the yellow end of the stellar spectrum.
It is a population 3 star, meaning that it has a high metallic content, a fact which we exploit to aid in our observations.

\subsection{Structure}
The sun is approximately $6.96 \times 10^{8}$ m in diameter and has an average density of $1.4$ gcm^{-3}, approximatley $40\%$ more than the density of water and constitutes $98\%$ of the mass within our solar system.

The structure of the Sun can be divided into internal and external.
Internal structure has been inferred by the use of techniques such as seismology, therefore, we still have many questions as to the exact mechanisms dominating below the photosphere.
These examinations have revealed a stratified structure, and the centre of which is the fusion core.
Currently, the fusion process is converting $2$ Hydrogen atoms to $1$ Deuteron, positron and neutrino; this Deuteron then reacts with another proton to form ${^3}He$ and a gamma particle, and lastly two $^{3}He$ combine to form $^{4}He$ and two protons, known as the proton-proton chain. 
However as the star evolves this process will change as one fuel source runs out, the previous product becomes the new fuel.
\emph{E.g.} the next phase would convert two Helium atoms to a Beryllium atom and a gamma particle.
The particles given off in the form of gamma particles and neutrinos carry away the excess energy, and go on to form the radiative zone.
These processes require extremely high pressure and temperature in order to overcome the binding energy of the atoms taking part in the process.
The conditions needed for this fusion process are generated by gravitational pressure exerted by the rest of the star which inherently increases the temperature in accordance with the ideal gas law.

The radiative zone is appropriately named, in that, the excess energy generated in the core radiates through this section.
It has been calculated that a photon emitted in the core takes on average $100,000$ years to move through the radiative zone as a result of the random walk, a process by which a photon is emitted and absorbed repeatedly.

The tachocline is a thin layer between the radiative zone and the convection zone, of which not a great deal is known.
This is the point where p-mode oscillations, used in seismology, cease to penetrate further into the solar interior and is strongly suspected to have a crucial role in generating the solar magnetic field.
There have been recent studies which suggest that the tacholine is responsible for several $1.3$ to $3$ year cycles in feature observed higher in the solar atmosphere, however this has not been definitively proven.
It has also been proposed as the source for a dynamo generating the magnetic field.
The most important result of the tachocline is that at this point, the motion changes from uniform behaviour of the radiation zone to the differential rotation of the convective zone.
This differential rotation is the cause of much of the complexity inherent in the solar body.

% more needed here
The layer above the tachocline is the convective zone.
Within this region, radiative heat transport ceases to be as effective, and as such, convection becomes the primary form energy transport.
An result of this lower temperature is a change in the state of the plasma.
Within the radiative zone the plasma is approximately $5$ MK and is consequently fully ionised, however with the decrease in temperature, comes a transition in the behaviour of the plasma, density drops, and the heavier elements are now not ionised.
The process by which convection takes place is cellular in nature, hot plasma rising and cool plasma falling back through the solar interior.
 


The photosphere is the layer of the Sun which we can observe with the naked eye.
The photosphere is so called, because this layer emits light in the visible spectrum, and has a temperature range of $6000$ K at the base, and $4700$ K at the top. 
From a distance, the photosphere appears to be a smooth sphere, however upon closer inspection we observe granulation, as a result of the convection zone, is ubiquitous throughout the layer.
Granulation appears as dark lines and bright 'bubbles' expanding and collapsing as hot material rises, cools and consequently falls back down into the solar interior.
These features tend to measure approximately $2$ Mm from one boundary to the other, which, are know as inter-granular lanes.
When granules bulk motion is observed , it becomes apparent that they group into larger, super-granules, where the overall motion of the granules radiates away from a central point until they fade at a point where they meet similar granules from another super-granule.
% need a granulation figure here

The most prominent feature in the photosphere are Sunspots, these are widely documented.
Observations of sunspots are dated as far back as Galileo and his first telescopes.
These features are significantly cooler that the surrounding photosphere, with a dark centre, known as the umbra, through which open magnetic field emerges from the solar interior.
Spreading away from the umbra is the pre-numbra, a ring of long thin structures spreading away from the umbra.
These have been widely studied and have become known as fibrils.
The reason these are so widely studied, is that it has become clear that waves travel up these features from the solar interior.
Sunspots are normally part of an extremely complex system of magnetic field, meaning that there are usually many in one active region.
It is not yet clear the exact mechanism by which sunspots are formed, however the favoured hypothesis is that of flux rope emergence.
In this model a 'rope' of magnetic field is forced up through the convection zone and the photosphere carrying plasma frozen into the magnetic field lines with them.
As a result of this formation mechanism, active regions have regions of both negative and positive polarity.


% Chromosphere
The first observations of the chromosphere were made during total solar eclipses.
When the moon entirely covered the solar disk \emph{i.e.} the photosphere, apparent at the solar limb were colourful streamers radiating from the solar centre.
The density drops sharply from the photosphere to the order of $10{-4}$ and is approximately $2$ Mm thick. 
Over this $2$ Mm layer the temperature rises from $4000$ K to $25,000$ K, this result is currently one of the primary focuses of the solar research community.
The chromosphere is incredibly complex interms of its magnetic structure and the transport of heat, consequently the chromosphere plays and essential role in the formation of explosive events such as solar flares and coronal mass ejections (CMEs).

It is in the chromosphere where we now see the magnetic fields which are rooted in the sunspots we observe in the photosphere.
These appear as large scale loops of plasma which has been locked into the magnetic field lines, extending through this region and right into the corona, they regularly reach 10's of Mm into the atmosphere.
Lower in the chromosphere, we find a second, smaller scale population of loops, and are a few Mm across on average.
Small scale chromospheric loops have been presented as a possible link between the photosphere and the chromosphere, due the their most likely formation being a small scale flux emergence event.
% more needs to be said here

Examining any image of the chromosphere, it is almost impossible to miss the fabled 'forest of spicules'.
Spicules are small, short-lived, explosive events with origins in the chromosphere which extend through the transition region and into the corona.
They are ubiquitous in the chromosphere, their number density being of the order of $10^5$ at any given time, however, their spatial distribution is not uniform.
They are found to form on intergranular lanes, as such their is much debate about how they form.
Candidates for the formation mechanism are; magnetic reconnection, p-mode driven and 'plasma drains', although there is no conclusive evidence for any of these drives.
Spicules high number density has lead many to propose them as a possible solution to the coronal heating problem, be that through instability or propagation of waves into the atmosphere directly.
Recent in depth study has revealed the possibility of two populations of spicules, dubbed, Type-I and Type-II.
Type-I have been defined to be longer lived and less explosive, they are also observed to emerge and have a ballistic motion away from and returning to the Sun.
However, Type-II show significantly more explosive velocities, up to $150$ km/s recorded during their initial formation, however, they are not observed to return and lifetimes are not expected to exceed $5$ mins. 

% need a section on the magnetic network here somewhere

% better citing needed here
Since these initial propositions, however, there has been doubt cast as to whether this is the case or not. 
Cite Zhang 2010 found no statistical separation of populations within spicules and recent publications but the original authors have shown that Type-2 spicules disappear from Ca II and reappear in the hotter Si IV line.
This would imply that spicules are heating as they accelerate through the atmosphere, whether this is because the underlying formation mechanism is different or there is sufficient energy in the initialisation of the spicule to cause heating as they propagate through the atmosphere, has yet to be made clear.

Particularly noticeable in the chromosphere is the appearance of coronal holes.
These appear as regions of dark amongst the bright chromospheric features, this is a result of the cool plasma, lower in the atmosphere becoming visible through these holes.
During the minimum phase of the solar cycle, there are usually two prominent coronal holes at the solar poles. 
These can cover half of the solar disk during particularly inactive solar minima, however, at solar maxima, these polar coronal holes disappear as the magnetic field becomes increasingly complex.
The coronal holes are characterised by open magnetic field lines extending up though the solar atmosphere, whereas, in the quiet Sun (areas not coronal holes) the magnetic field lines are closed, generally forming small and large scale loops.

This magnetic field configuration of open field lines at the poles and profound non-uniformity between is a result of the differential rotation above the tachocline.
Due to faster angular velocities at the equator than at the poles, magnetic field lines which would be straight, pole to pole, are warped in accordance with the frozen in condition.
Eventually, the magnetic field between $60^\circ$ and $-60^\circ$ forms into approximate bands of alternating opposing polarity magnetic field.
The mechanism by which this structure forms, results in converging bands, \emph{i.e.} a sideways 'V' symmetrical around the equator.
This can act as a guide for other solar features, such as sunspots, pushing them towards the solar equator from either side.
% not finished with the chromosphere yet 

There are several large scale features with their roots in the chromosphere.
Filaments, and their off limb counterparts prominences, are very prevalent in solar imaging.
Filaments are observed as dark strands of plasma, having risen from the cool photosphere against the hotter chromosphere, that can extend far across the solar disk and have lifetimes of many months.
Prominences are the same features observed over the limb, and hence appear as bright features against the dark sky.
This allows us to observe the impact on the atmosphere more closely. 
Upon inspection, we observe the plasma falling away from the main body of the prominence, becoming known as coronal rain.
The effect this rain has on the atmosphere is not yet fully understood, but there has be disussion as to its merits in terms of possible heating or cooling effects.

At the top of the chromosphere, temperature of the plasma increases rapidly over a very short distance, approximately $500$ km, this is known at the transition region.
It acts as a barrier, and amplifier, between chromosphere and corona with the temperature rising from $25,000$ K to $2$ MK, the mechanism which causes this is still not understood and is one of the prominent problems in solar physics.


Corona

Solar Wind

Heliopause

\subsection{History of observations (generally)}

\section{Plasma behaviour}

\subsection{MHD}

\subsection{Reconnection}

\section{Jet and Macrospicules}

\subsection{The ejecta zoo}

\subsection{Macrospicules}

\subsection{Numerical Simulations}
































\begin{pycode}[chapter1]
from __future__ import print_function

ch1 = texfigure.Manager(pytex, number=1, base_path='./Chapter1/')
\end{pycode}

\begin{pycode}[chapter1]
fig = plt.figure(dpi=100, figsize=texfigure.figsize(pytex, height_ratio=1.5))

xx = np.linspace(0, 6*np.pi, 1000)
plt.plot(xx, np.sin(xx))

fig.tight_layout()
fig.subplots_adjust(bottom=0.15)
photosphere = ch1.save_figure('sin', fig, fext='.pgf')
photosphere.caption = "Some sine wave."
\end{pycode}

\py[chapter1]|photosphere|



Hello world, checkout \cref{fig:sin}.


\begin{pycode}[chapter1]
	
multi = texfigure.MultiFigure(4,1)
width = 0.9

X = [[5,6], [7,8], [5,6], [7,8]]
Y = [[1,2], [3,4], [1,2], [3,4]]

for x,y in zip(X,Y):
	fig = plt.figure(figsize=texfigure.figsize(pytex, scale=width))
	plt.plot(x, y, 'o')
	
	Fig1 = ch1.save_figure('test', fig)
	Fig1.subfig_width=r"{}\columnwidth".format(width)
	multi.append(Fig1)
\end{pycode}

\py[chapter1]|multi[0:2]|
\py[chapter1]|multi[2:4]|