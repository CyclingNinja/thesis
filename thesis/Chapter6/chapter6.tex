% !TeX root = ../thesis.tex
%*****************************************************************************************
%*********************************** Second Chapter ***************************************
%*****************************************************************************************

\label{ch:conc}
\chapter{Conclusions}

Macrospicules are an intriguing feature of the solar atmosphere.
The work that is presented in this thesis has attempted to shed some light on these prevalent parts of the atmosphere.
\cref{ch:3} can be easily compared to the previous works on a like-for-like basis.
The previous examples of measurements, given by \cite{Bohlin1975} and \cite{Dere89}, are compared in \cref{table:final_properties}.
The values found here compare particularly well with Bohlin, however, the work by Dere finds more conservative values.
Given the improved nature of instrumentation and measuring tools utilised in this work, we assert that the values found here should become the new expected measurements defining a macrospicule.

This study also found no divergence in the population of features measured.
This might be expected in the case where some of the macrospicules are the cool component of a hotter feature.
Of course, the possibility remains that, they are part of a larger scale feature and the properties are similar to those of regular macrospicules, but there is no evidence to support this hypothesis within this study.   

With respect to the ballistics and enegetic of the feature, we find the unexpected effect of a macrospicule becoming thinner post apex of its motion.
Another theory could be that the magnetic pressure would cause the width to increase, however the evidence does not agree with this, as such we need to think about the macrospicule not just as a column of regular plasma.
The likely case is that the plasma begins to flow down the magnetic field lines, in a similar manner to evanescence, resulting in a decrease in the width of the feature.

\cref{ch:5} is the culmination of this thesis and is the most comprehensive view of a macrospicule in this work.
The most significant result of which, is a demonstrated very clearly in the line of sight dopplergrams.
The macrospicule feature clearly shows a rotation initially in the anti-clockwise direction, before slowing in rate, stopping and then rotating clockwise.
It is also apparent that the feature 'unwinds' from the tip of he macrospicule downwards before developing across the entire feature.  
This unequivocally proves that macrospicules are have an intrinsic magnetic field.
This is interesting, as it has implications on the formation mechanism.
If it were the case that these were merely an expulsion of plasma from a reconnection method highlighed in \cref{sec:recon}, a more complex process is underway.

Fortunately, the resolution of SST allows us to examine the footpoint of the macrospicule.
The conclusion that arises from this is that the macrospicule seems to undergo a formation event similar to Shibata's standard jet formation model.
The brightenings evident in the H$\alpha$, show too much smaller brightpoints evolving and joining before the spire eventually forms.
The extension of this feature in the lower temperature emission lines, where it extends the furthest, is roughly $12$ Mm, not large for a macrospicule.
And so, the fact that this small scale jet features a very similar formation profile to larger coronal jets.
The implications for classification are that the waters are now muddied.
Should macrospicules be classed as standard jets?
Given that this feature is very thin, but the first work shows average widths of ~$7$ Mm, do we also get 'blowout' style macrospicules?
The hope is that this work can now be built on, and a more general classification for jet like features based on the underlying physics, generating these features.  
 


