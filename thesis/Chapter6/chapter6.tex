% !TeX root = ../thesis.tex
%*****************************************************************************************
%*********************************** Conclusion ***************************************
%*****************************************************************************************

\label{ch:conc}
\chapter{Conclusions}

Macrospicules are an intriguing feature of the solar atmosphere.
The work that is presented in this thesis has attempted to shed some light on these prevalent parts of the atmosphere.
\cref{ch:3} can be easily compared to the previous works on a like-for-like basis.
The previous examples of measurements, given by \cite{Bohlin1975} and \cite{Dere89}, are compared in \cref{table:final-properties}.
The values found here compare particularly well with Bohlin, however, the work by Dere finds more conservative values.
Given the improved nature of instrumentation and measuring tools utilised in this work, we assert that the values found here should become the new expected measurements defining a macrospicule.

This study also found no divergence in the population of features measured.
Such an effect could be expected were MS very similar to another feature, but with separate underlying physics.
Of course, the possibility remains that, they are part of a larger scale feature, such as X-Ray jets, which, has been discussed on multiple occasions, \cite{Parenti2002, Kamio2010}.
In these cases, the jet-like features' properties are similar to those of regular MS, but there is no evidence to support this hypothesis within this study.   

With respect to the ballistics and energetics of the feature, we find the unexpected effect of a macrospicule becoming thinner as it falls down in the atmosphere.
Another theory could be that the magnetic pressure would cause the width to increase, however the evidence does not agree with this, as such we need to think about the macrospicule not just as a column of regular plasma.
The likely case is that the plasma begins to flow down the magnetic field lines, in a similar manner to evanescence, resulting in a decrease in the width of the feature.

\cref{ch:4} demonstrates that macrospicules' properties are dictated by larger processes in the Sun.
That macrospicules tend to form along an active longitude is likely due to the influence of the tachocline.
This result has subsequently been given more credence as a result of the work undertaken by \cite{Kiss2017} finding a short term, $1$ - $2$ year cycle in macrospicule length.
Again, this could be due to the influence of the tachocline dynamo effects creating the global magnetic field.

\cref{ch:5} presents the most comprehensive view of a macrospicule in this thesis.
The most significant result, is demonstrated very clearly in the line of sight dopplergrams.
The macrospicule feature clearly shows a rotation initially in the anti-clockwise direction, before slowing in rotation rate, stopping and then rotating clockwise.
This is the first observation of such behaviour in a macrospicule.
In contrast, there have been multiple observations and simulations of rotating jet-like features, e.g. \cite{Kamio2010, Curdt2011, Majarska2011}, however, this is the first work demonstrating torsional oscillating motion in either jet or macrospicules.
It is also apparent that the observed `unwinding' motion starts from the tip of the macrospicule before moving downwards.  
This unequivocally proves that MS have an intrinsic magnetic field.
The presence of a magnetic field in the macrospicule constrains the formation mechanism, to those involving the simultaneous expulsion of plasma and magnetic field, which is highlighted in \cref{sec:recon}.

The resolution of SST allows us to examine the footpoint of the macrospicule.
The conclusion that arises from this is that the macrospicule seems to undergo a formation event similar to the \cite{Shibata1992} standard jet formation model.
The brightenings evident in the H$\alpha$ line, show two much smaller brightpoints evolving and joining before the spire eventually forms.
The extension of this feature in the lower temperature emission lines, where it extends the furthest, is roughly $12$ Mm.
We have therefore observed a small-scale feature with a very similar formation mechanism to larger coronal jets.
This leads to the conclusion that macrospicules should perhaps be classified as standard jets.
Given that this feature is very thin, but \cref{ch:3} shows average widths of the order of $7$ Mm, therefore should the population be separated in terms of formation processes, i.e. standard and `blowout' style MS, in the same way as jets?

Macrospicules and their relationship (or lack thereof) with jets, will continue to be essential to our understanding of the solar atmosphere, given that they originate in the low chromosphere and extend up into the corona.
The questions of the most importance at the moment are of their structure and formation mechanism.
Both of these may be answered by the new DKIST telescope.
The leap forward that DKIST will take with respect to spatial and temporal resolution will allow us to examine the footpoints of these features with significantly greater accuracy.
Possibly more intriguing is the body of MS themselves.
Given the `unwinding' of magnetic twist observed in \cref{ch:5}, it may be possible to inspect individual strands of flux rope, which could answer:

\begin{itemize}
	\item{Are the MS unwinding in all cases?}
	\item{Could the twisting motion be a result of a driver from below?}
	\item{Do the MS always have a torsional or helical component?}
	\item{Can MS be considered as a major factor with respect to heating or solar wind acceleration?}
\end{itemize}
 
 
Those are just the questions considering MS on their own.
Macrospicules' current place in the plethora of jet-like solar features is still far from clear, despite this work.
Are MS instances of multiple spicules `superposing' and becoming a much larger feature as supposed by \cite{Xia2005}?
Are they the cool component of X-Ray jets as in \cite{Parenti2002}? In which case should we merely be calling them X-Ray jets observed in chromospheric emission lines?
There has been much work on the subject of H$\alpha$ and He II $30.4$ nm MS, including work by \cite{LaBonte79,Wang1998} but we have yet to conclusively prove that these features are one and the same.
However, \cref{ch:5}, again demonstrates that this is the case.
 
The future of the study of macrospicules lies with observations.
At the moment, there is no shortage of models sufficiently complex to describe, in detail, the method by which macrospicules are formed, authors such as \cite{Murawski2011}, \cite{Archontis2005} and \cite{Moreno2008} have covered his extensively.
What is required is tuning of the initial conditions and magnetic configuration at the time of onset.

This is now possible with the building of DKIST.
DKIST will be the most advanced telescope of the current age of solar observations, with a 4 meter mirror able to resolve $10$ km on the solar surface and a very high temporal cadence, it can be pushed as low as $1/3$ of a second [\cite{Woeger2016}].
With this instrument, we will be able to record the moment that a macrospicule is formed in extreme detail.
The most efficient way, given macrospicules infrequency compared to spicules, would be to focus attention to the coronal hole boundaries given that $20\%$ of macrospiules are formed there, \cref{ch:3}, and the area too cover is much smaller than the alternative coronal holes and quiet Sun.

Over a large sample window, collecting as many macrospicules evolution as possible, would allow a statistical distribution of the rotational velocities of the macrospicules to be found.
Testing whether the rotational velocity measured in \cref{ch:5} is the expected value or overly slow.
This will have an impact going forward as it would demonstrate whether the Kelvin-Helmholtz instability [\cite{Zaqarashvili2014}] could heat the corona through the volume of spicules/macrospicules in the chromosphere. 

However, the most import subject with respect to macrospicules is their classification among the solar jets. 
\cref{ch:5} demonstrates that a feature with similar size and dimension to those in \cref{ch:3} can have formation patterns equatable to much larger features such as X-Ray and EUV jets.
As the study of jets carries forward, the question at the front the solar physicist's mind should be, are jets observed in the lower temperatures actually different features to hotter jets, or just their footprint lower in the atmosphere?


 


